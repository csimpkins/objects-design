\documentclass{beamer}

\newcommand{\course}{CS 2340 Objects and Design}
\newcommand{\lesson}{Git for Teams}
\newcommand{\code}{http://www.cc.gatech.edu/~simpkins/teaching/gatech/cs2340/code}

\author[Chris Simpkins] 
{Christopher Simpkins \\\texttt{chris.simpkins@gatech.edu}}
\institute[Georgia Tech] % (optional, but mostly needed)

\date[CS 1331]{}

\subject{\lesson}


% If you have a file called "university-logo-filename.xxx", where xxx
% is a graphic format that can be processed by latex or pdflatex,
% resp., then you can add a logo as follows:

% \pgfdeclareimage[width=0.6in]{coc-logo}{cc_2012_logo}
% \logo{\pgfuseimage{coc-logo}}

\mode<presentation>
{
  \usetheme{Berlin}
  \useoutertheme{infolines}

  % or ...

 \setbeamercovered{transparent}
  % or whatever (possibly just delete it)
}

\usepackage{hyperref}
\usepackage{fancybox}
\usepackage{listings}
\usepackage[abbr]{harvard}

\usepackage[english]{babel}
% or whatever

\usepackage[latin1]{inputenc}
% or whatever

\usepackage{times}
\usepackage[T1]{fontenc}
% Or whatever. Note that the encoding and the font should match. If T1
% does not look nice, try deleting the line with the fontenc.


\usepackage{listings}
 
% "define" Scala
\lstdefinelanguage{scala}{
  morekeywords={abstract,case,catch,class,def,%
    do,else,extends,false,final,finally,%
    for,if,implicit,import,match,mixin,%
    new,null,object,override,package,%
    private,protected,requires,return,sealed,%
    super,this,throw,trait,true,try,%
    type,val,var,while,with,yield},
  otherkeywords={=>,<-,<\%,<:,>:,\#,@},
  sensitive=true,
  morecomment=[l]{//},
  morecomment=[n]{/*}{*/},
  morestring=[b]",
  morestring=[b]',
  morestring=[b]""",
}

\usepackage{color}
\definecolor{dkgreen}{rgb}{0,0.6,0}
\definecolor{gray}{rgb}{0.5,0.5,0.5}
\definecolor{mauve}{rgb}{0.58,0,0.82}
 
% Default settings for code listings
\lstset{frame=tb,
  language=scala,
  aboveskip=2mm,
  belowskip=2mm,
  showstringspaces=false,
  columns=flexible,
  basicstyle={\scriptsize\ttfamily},
  numbers=none,
  numberstyle=\tiny\color{gray},
  keywordstyle=\color{blue},
  commentstyle=\color{dkgreen},
  stringstyle=\color{mauve},
  frame=single,
  breaklines=true,
  breakatwhitespace=true,
  keepspaces=true
  %tabsize=3
}


\title[\course] % (optional, use only with long
                                      % paper titles)
{\lesson}

\subtitle{}
%% {Include Only If Paper Has a Subtitle}

\newcommand{\link}[2]{\href{#1}{\textcolor{blue}{\underline{#2}}}}


% \beamerdefaultoverlayspecification{<+->}


\begin{document}

\begin{frame}
  \titlepage
\end{frame}




%------------------------------------------------------------------------
\begin{frame}[fragile]{Git for Teams}


\begin{itemize}
\item Tagging
\item Branching
\item Distributed Workflow
\end{itemize}


\end{frame}
%------------------------------------------------------------------------

%------------------------------------------------------------------------
\begin{frame}[fragile]{Tagging}


A tag is an alias for a commit.  Create an annotated tag with {\tt git tag -a}
\begin{lstlisting}[language=bash]
[chris@nijinsky ~/work/vcs/github/tomcat-todo]
$ git tag -a m1 -m "Milestone 1: initial demo to class."
\end{lstlisting}
This creates an annotated tag ({\tt -a}) called {\tt m1} with a tag message of  "Milestone 1: initial demo to class."  We can list tags:
\begin{lstlisting}[language=bash]
[chris@nijinsky ~/work/vcs/github/tomcat-todo]
$ git tag
m1
\end{lstlisting}
And, for annotated tags, we can show the tag's details:
\begin{lstlisting}[language=bash]
[chris@nijinsky ~/work/vcs/github/tomcat-todo]
$ git show m1
tag m1
Tagger: Chris Simpkins <chris.simpkins@gmail.com>
Date:   Wed Jun 12 07:14:17 2013 -0400

Milestone 1: initial demo to class.
...
\end{lstlisting}

\end{frame}
%------------------------------------------------------------------------

%------------------------------------------------------------------------
\begin{frame}[fragile]{Reverting to a Tag and Sharing Tags}


Lets say we add and commit {\tt src/main/webapp/stylesheets/main.css} and modify {\tt src/main/webapp/list.jsp} to use the stylesheet, but then we want to revert our working copy of the repo to the {\tt m1} tag.  We can do this with {\tt git checkout}
\vspace{-.05in}
\begin{lstlisting}[language=bash]
$ git checkout m1
\end{lstlisting}
Git gives us a message about being in a 'detached HEAD' state, and the new stylesheet changes are gone from our working copy.

We can get back to the current version of the rep by checking out master:
\begin{lstlisting}[language=bash]
$ git checkout master
\end{lstlisting}
\vspace{-.05in}
Tags aren't pushed to a remote unless you tell git to push the tags.  You can push all of your tags with {\tt git push --tags} or a specific tag with {\tt git push tag m1}.  You'll need to do this when you tag your milestones.

\end{frame}
%------------------------------------------------------------------------

%% %------------------------------------------------------------------------
%% \begin{frame}[fragile]{}


%% \begin{lstlisting}[language=bash]

%% \end{lstlisting}


%% \end{frame}
%% %------------------------------------------------------------------------


%% %------------------------------------------------------------------------
%% \begin{frame}[fragile]{Branching}


%% First: read Pro Git Chapter 4 to get the concepts, which we don't have time to cover in detail here.  We're just covering the basics you'll need to do your group project.\\

%% A branch is just a movable pointer to a commit.  Each repo has a {\tt master} branch by default.  Create a new branch with {\tt git branch}.  For example, say we have a task of 
%% \begin{lstlisting}[language=bash]

%% \end{lstlisting}


%% \end{frame}
%% %------------------------------------------------------------------------

%% %------------------------------------------------------------------------
%% \begin{frame}[fragile]{}


%% \begin{lstlisting}[language=bash]

%% \end{lstlisting}


%% \end{frame}
%% %------------------------------------------------------------------------

%% %------------------------------------------------------------------------
%% \begin{frame}[fragile]{}


%% \begin{lstlisting}[language=bash]

%% \end{lstlisting}


%% \end{frame}
%% %------------------------------------------------------------------------

%% %------------------------------------------------------------------------
%% \begin{frame}[fragile]{}


%% \begin{lstlisting}[language=bash]

%% \end{lstlisting}


%% \end{frame}
%% %------------------------------------------------------------------------

%% %------------------------------------------------------------------------
%% \begin{frame}[fragile]{}


%% \begin{lstlisting}[language=bash]

%% \end{lstlisting}


%% \end{frame}
%% %------------------------------------------------------------------------

% %------------------------------------------------------------------------
% \begin{frame}[fragile]{}


% \begin{lstlisting}[language=Java]

% \end{lstlisting}

% \begin{itemize}
% \item
% \end{itemize}


% \end{frame}
% %------------------------------------------------------------------------


\end{document}
