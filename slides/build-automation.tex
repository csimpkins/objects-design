\documentclass{beamer}

\newcommand{\course}{CS 2340 Objects and Design}
\newcommand{\lesson}{Build Automation}
\newcommand{\code}{http://www.cc.gatech.edu/~simpkins/teaching/gatech/cs2340/code}

\author[Chris Simpkins]
{Christopher Simpkins \\\texttt{chris.simpkins@gatech.edu}}
\institute[Georgia Tech] % (optional, but mostly needed)

\date[CS 1331]{}

\include{beamer-common}

% \beamerdefaultoverlayspecification{<+->}


\begin{document}

\begin{frame}
  \titlepage
\end{frame}


%------------------------------------------------------------------------
\begin{frame}[fragile]{Ant}

\vspace{-.07in}
\link{http://ant.apache.org/}{Ant} is a build automation tool, like {\tt make}.  Install it however you want (like with homebrew on a Mac), then put this in a file named {\tt build.xml} in the root directory of your blackjack project:
\vspace{-.07in}
\begin{lstlisting}[language=xml]
<?xml version="1.0" encoding="UTF-8"?>
<project name="blackjack" default="default" basedir=".">
  <path id="classpath">
    <fileset dir="target/classes" includes="**/*.class"/>
  </path>

  <target name="init">
    <mkdir dir="target"/>
    <mkdir dir="target/classes"/>
  </target>

  <target name="compile" depends="init">
    <javac srcdir="src/main/java"
           destdir="target/classes"
           classpathref="classpath"
           source="1.7"
           target="1.7" />
  </target>
</project>
\end{lstlisting}

\begin{itemize}
\item
\end{itemize}


\end{frame}
%------------------------------------------------------------------------

%------------------------------------------------------------------------
\begin{frame}[fragile]{Compiling with Ant}


Invoke the {\tt compile} target to compile the project:
\begin{lstlisting}[language=Java]
$ ant compile
Buildfile: /Users/chris/scratch/blackjack/build.xml

init:
...
compile:
...
BUILD SUCCESSFUL
Total time: 0 seconds
\end{lstlisting}

This will produce class files in {\tt target/classes}.  How would you run the {\tt Blackjack} class?

\end{frame}
%------------------------------------------------------------------------


%------------------------------------------------------------------------
\begin{frame}[fragile]{Runnable Jar Files}


The hard way: \link{http://docs.oracle.com/javase/tutorial/deployment/jar/appman.html}{Oracle's Jar docs}\\
\vspace{.1in}
The easy way: add this to your {\tt build.xml}:
\begin{lstlisting}[language=xml]
  <target name="package" depends="compile">
    <jar destfile="target/blackjack.jar">
      <fileset dir="target/classes"/>
      <manifest>
        <attribute name="Main-class"
                   value="edu.gatech.cs2340.blackjack.Blackjack"/>
      </manifest>
    </jar>
  </target>
\end{lstlisting}

Then you can do:

\begin{lstlisting}[language=bash]
$ ant package
...
BUILD SUCCESSFUL
Total time: 0 seconds
$ java -jar target/blackjack.jar
\end{lstlisting}



\end{frame}
%------------------------------------------------------------------------


%------------------------------------------------------------------------
\begin{frame}[fragile]{Generating Documentation}

Add this target to your {\tt build.xml}
\begin{lstlisting}[language=Java]
<target name="javadoc" description="Generate Javadoc" depends="compile">
    <javadoc destdir="target/docs/api"
             classpathref="classpath"
             access="private"
             version="true"
             use="true"
             author="true"
             overview="src/main/java/overview.html">
      <fileset dir="src/main/java" defaultexcludes="yes">
        <include name="**/*.java" />
      </fileset>
      <doctitle>
        <![CDATA[<h1>Blackjack API Documentation</h1>]]>
      </doctitle>
      <bottom>
        <![CDATA[<i>Copyright &#169; Georgia Tech. All Rights Reserved.</i>]]>
      </bottom>

      <link href="http://docs.oracle.com/javase/7/docs/api/"/>
    </javadoc>
  </target>
\end{lstlisting}



\end{frame}
%------------------------------------------------------------------------

% %------------------------------------------------------------------------
% \begin{frame}[fragile]{}

% \begin{itemize}
% \item
% \end{itemize}

% \begin{lstlisting}[language=Java]

% \end{lstlisting}



% \end{frame}
% %------------------------------------------------------------------------


\end{document}
