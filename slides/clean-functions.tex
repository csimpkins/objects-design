\documentclass{beamer}

\newcommand{\course}{CS 2340 Objects and Design}
\newcommand{\lesson}{Clean Functions}
\newcommand{\code}{http://www.cc.gatech.edu/~simpkins/teaching/gatech/cs2340/code}

\author[Chris Simpkins] 
{Christopher Simpkins \\\texttt{chris.simpkins@gatech.edu}}
\institute[Georgia Tech] % (optional, but mostly needed)

\date[CS 1331]{}

\subject{\lesson}


% If you have a file called "university-logo-filename.xxx", where xxx
% is a graphic format that can be processed by latex or pdflatex,
% resp., then you can add a logo as follows:

% \pgfdeclareimage[width=0.6in]{coc-logo}{cc_2012_logo}
% \logo{\pgfuseimage{coc-logo}}

\mode<presentation>
{
  \usetheme{Berlin}
  \useoutertheme{infolines}

  % or ...

 \setbeamercovered{transparent}
  % or whatever (possibly just delete it)
}

\usepackage{hyperref}
\usepackage{fancybox}
\usepackage{listings}
\usepackage[abbr]{harvard}

\usepackage[english]{babel}
% or whatever

\usepackage[latin1]{inputenc}
% or whatever

\usepackage{times}
\usepackage[T1]{fontenc}
% Or whatever. Note that the encoding and the font should match. If T1
% does not look nice, try deleting the line with the fontenc.


\usepackage{listings}
 
% "define" Scala
\lstdefinelanguage{scala}{
  morekeywords={abstract,case,catch,class,def,%
    do,else,extends,false,final,finally,%
    for,if,implicit,import,match,mixin,%
    new,null,object,override,package,%
    private,protected,requires,return,sealed,%
    super,this,throw,trait,true,try,%
    type,val,var,while,with,yield},
  otherkeywords={=>,<-,<\%,<:,>:,\#,@},
  sensitive=true,
  morecomment=[l]{//},
  morecomment=[n]{/*}{*/},
  morestring=[b]",
  morestring=[b]',
  morestring=[b]""",
}

\usepackage{color}
\definecolor{dkgreen}{rgb}{0,0.6,0}
\definecolor{gray}{rgb}{0.5,0.5,0.5}
\definecolor{mauve}{rgb}{0.58,0,0.82}
 
% Default settings for code listings
\lstset{frame=tb,
  language=scala,
  aboveskip=2mm,
  belowskip=2mm,
  showstringspaces=false,
  columns=flexible,
  basicstyle={\scriptsize\ttfamily},
  numbers=none,
  numberstyle=\tiny\color{gray},
  keywordstyle=\color{blue},
  commentstyle=\color{dkgreen},
  stringstyle=\color{mauve},
  frame=single,
  breaklines=true,
  breakatwhitespace=true,
  keepspaces=true
  %tabsize=3
}


\title[\course] % (optional, use only with long
                                      % paper titles)
{\lesson}

\subtitle{}
%% {Include Only If Paper Has a Subtitle}

\newcommand{\link}[2]{\href{#1}{\textcolor{blue}{\underline{#2}}}}


% \beamerdefaultoverlayspecification{<+->}


\begin{document}

\begin{frame}
  \titlepage
\end{frame}

%------------------------------------------------------------------------
\begin{frame}[fragile]{Functions Should be Small and Do one Thing Only}


\begin{itemize}
\item The first rule of functions: functions should be small.
\item The second rule of functions: functions should be small.
\end{itemize}
How small?  A few lines, 5 or 10.  ``A screen-full'' is no longer meaningful with large monitors and small fonts.

Some signs a function is doing too much:
\begin{itemize}
\item Many parameters.  ``If you have a procedure with ten parameters, you probably missed some.'' -- Perlis
\item ``Sections'' within a function, often delimited by blank lines.
\item Deeply nested logic.
\end{itemize}

\end{frame}
%------------------------------------------------------------------------

%------------------------------------------------------------------------
\begin{frame}[fragile]{Writing Functions that Do One Thing}


\begin{itemize}
\item One level of abstraction per function.
\begin{itemize}
\item A function that implements a higher-level algorithm should call helper functions to execute the steps of the algorithm.
\end{itemize}
\item Write code using the stepdown rule.
\begin{itemize}
\item Code should read like a narrative from top to bottom.
\item Read a higher level function to get the big picture, the functions below it to get the details.
\end{itemize}

\end{itemize}


\end{frame}
%------------------------------------------------------------------------

%------------------------------------------------------------------------
\begin{frame}[fragile]{Switch Statements}


Switch statements do more than one thing by design.  Consider:
\begin{lstlisting}[language=Java]
public class Payroll
  public Money calculatePay(Employee e) throws InvalidEmployeeType {
    switch (e.type) {
      case COMMISSIONED:
        return calculateCommissionedPay(e);
      case HOURLY:
        return calculateHourlyPay(e);
      case SALARIED:
        return calculateSalariedPay(e);
      default:
        throw new InvalidEmployeeType(e.type); }
  }
}
\end{lstlisting}

\begin{itemize}
\item This class violates Single Responsibility Principle because there are multiple reasons to change it (payroll, employee).
\item This class violates the Open Closed Principle becuase extending the system to handle new types of employees requires changing the code in Payroll.
\end{itemize}


\end{frame}
%------------------------------------------------------------------------

%------------------------------------------------------------------------
\begin{frame}[fragile]{Replacing {\tt switch} Statements with Polymorphism}


\begin{lstlisting}[language=Java]
public abstract class Employee {
  public abstract boolean isPayday();
  public abstract Money calculatePay();
  public abstract void deliverPay(Money pay);
}
public class EmployeeFactory {
  public Employee makeEmployee(EmployeeRecord r) throws InvalidEmployeeType { 
    switch (r.type) {
      case HOURLY:
        return new HourlyEmployee(r);
      case SALARIED:
        return new SalariedEmploye(r);
      default:
        throw new InvalidEmployeeType(r.type);
    }
  }
}
\end{lstlisting}

\begin{itemize}
\item When we add a new Employee class, we only need to change the factory.
\end{itemize}


\end{frame}
%------------------------------------------------------------------------

%------------------------------------------------------------------------
\begin{frame}[fragile]{Function Parameters}


Common monadic (one argument) forms
\begin{itemize}
\item Predicate functions: {\tt boolean fileExists(``MyFile'')}
\item Transformations: {\tt InputStream fileOpen(``MyFile'')}
\item Events: {\tt void passwordAttemptFailedNtimes(int attempts)}
\end{itemize}

Dyadic, triadic, and higher numbers of function arguments are much harder to get right.  Even one argument functions are problematic.  Consider flag argumets:
\begin{itemize}
\item Instead of {\tt render(boolean isSuite)}, a call to which would look like {\tt render(true)}, 
\item write two methods, like {\tt renderForSuite()} and {\tt renderForSingleTest()}
\end{itemize}

Keep in mind that in OOP, every instance method call has an implicit argument: the object on which it is invoked.
\end{frame}
%------------------------------------------------------------------------

%------------------------------------------------------------------------
\begin{frame}[fragile]{Minimizing the Number of Arguments}


Use objects.  Instead of
\begin{lstlisting}[language=Java]
public void doSomethingWithEmployee(String name, double pay, Date hireDate)
\end{lstlisting}
Represent emplyees with a class:
\begin{lstlisting}[language=Java]
public void doSomethingWith(Employee employee)
\end{lstlisting}

Use argument lists
\begin{lstlisting}[language=Java]
public int max(int ... numbers)
public String format(String format, Object... args)
\end{lstlisting}


\end{frame}
%------------------------------------------------------------------------

%------------------------------------------------------------------------
\begin{frame}[fragile]{Have no Side Effects}


The checkPassword method below:
\begin{lstlisting}[language=Java]
public class UserValidator {
  private Cryptographer cryptographer;
  public boolean checkPassword(String userName, String password) {
    User user =  UserGateway.findByName(userName);
    if (user != User.NULL) {
      String codedPhrase = user.getPhraseEncodedByPassword();
      String phrase = cryptographer.decrypt(codedPhrase, password);
      if ("Valid Password".equals(phrase)) {
        Session.initialize();
        return true; }
    }
    return false; }
}
\end{lstlisting}
Has the side effect of initializing the session.
\begin{itemize}
\item Might erase an existing session, or
\item might create temporal coupling: can only check password for user that doesn't have an existing session.
\end{itemize}


\end{frame}
%------------------------------------------------------------------------

%------------------------------------------------------------------------
\begin{frame}[fragile]{Output Arguments}


Arguments are naturally interpreted as inputs.  Avoid using them as outputs, as in:
\begin{lstlisting}[language=Java]
public void appendFooter(StringBuffer report)
\end{lstlisting}
A call to this method would look like {\tt appendFooter(s);} and you'd need to read the method signature to figure out what was going on.  Better to have a function return a value or mutate an object:
\begin{lstlisting}[language=Java]
report.appendFooter();
\end{lstlisting}
or
\begin{lstlisting}[language=Java]
String footer = generateFooter();
report.appendFooter(footer);
\end{lstlisting}

\end{frame}
%------------------------------------------------------------------------

%------------------------------------------------------------------------
\begin{frame}[fragile]{Command Query Separation}


Consider:
\begin{lstlisting}[language=Java]
public boolean set(String attribute, String value);
\end{lstlisting}
We're setting values and querying ... something, leading to very bad idioms like
\begin{lstlisting}[language=Java]
if (set("username", "unclebob"))...
\end{lstlisting}
Better to separate commands from queries:
\begin{lstlisting}[language=Java]
if (attributeExists("username")) {
  setAttribute("username", "unclebob");
  ...
}
\end{lstlisting}


\end{frame}
%------------------------------------------------------------------------

%------------------------------------------------------------------------
\begin{frame}[fragile]{Prefer Exceptions to Error Codes}

\vspace{-.06in}
Returning error codes forces client programmer to mix error handling with main logic (and often leads to ugly nested code):
\vspace{-.05in}
\begin{lstlisting}[language=Java]
if (deletePage(page) == E_OK) {
  if (registry.deleteReference(page.name) == E_OK) {
    if (configKeys.deleteKey(page.name.makeKey()) == E_OK){
      logger.log("page deleted");
    } else {
      logger.log("configKey not deleted");
    }
  } else {
    logger.log("deleteReference from registry failed"); }
} else {
  logger.log("delete failed"); return E_ERROR;
}
\end{lstlisting}
\vspace{-.06in}
Let language features help you:
\vspace{-.05in}
\begin{lstlisting}[language=Java]
try {
  deletePage(page);
  registry.deleteReference(page.name);
  configKeys.deleteKey(page.name.makeKey());
} catch (Exception e) {
  logger.log(e.getMessage());
}
\end{lstlisting}

\end{frame}
%------------------------------------------------------------------------

%------------------------------------------------------------------------
\begin{frame}[fragile]{Extract Try/Catch Blocks}


You can make your code even clearer by extracting try/catch statements into functoins of their own:
\begin{lstlisting}[language=Java]
public void delete(Page page) {
  try {
    deletePageAndAllReferences(page); }
  catch (Exception e) {
    logError(e);
  }
}
private void deletePageAndAllReferences(Page page) throws Exception {
  deletePage(page);
  registry.deleteReference(page.name);
  configKeys.deleteKey(page.name.makeKey());
}
private void logError(Exception e) {
  logger.log(e.getMessage());
}
\end{lstlisting}


\end{frame}
%------------------------------------------------------------------------

%------------------------------------------------------------------------
\begin{frame}[fragile]{A Few More Function Writing Tips}


\begin{itemize}
\item Don't repeat yourself.  Extract oft-used code into functions.
\item Don't shackle yourself with structured programming.  Sometimes multiple returns or even -- gasp! -- {\tt break} and {\tt continue} lead to clearer code. 
\end{itemize}



\end{frame}
%------------------------------------------------------------------------


% %------------------------------------------------------------------------
% \begin{frame}[fragile]{}


% \begin{lstlisting}[language=Java]

% \end{lstlisting}

% \begin{itemize}
% \item
% \end{itemize}


% \end{frame}
% %------------------------------------------------------------------------


\end{document}
