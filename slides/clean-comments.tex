\documentclass{beamer}

\newcommand{\course}{CS 2340 Objects and Design}
\newcommand{\lesson}{Clean Comments}
\newcommand{\code}{http://www.cc.gatech.edu/~simpkins/teaching/gatech/cs2340/code}

\author[Chris Simpkins] 
{Christopher Simpkins \\\texttt{chris.simpkins@gatech.edu}}
\institute[Georgia Tech] % (optional, but mostly needed)

\date[CS 1331]{}

\subject{\lesson}


% If you have a file called "university-logo-filename.xxx", where xxx
% is a graphic format that can be processed by latex or pdflatex,
% resp., then you can add a logo as follows:

% \pgfdeclareimage[width=0.6in]{coc-logo}{cc_2012_logo}
% \logo{\pgfuseimage{coc-logo}}

\mode<presentation>
{
  \usetheme{Berlin}
  \useoutertheme{infolines}

  % or ...

 \setbeamercovered{transparent}
  % or whatever (possibly just delete it)
}

\usepackage{hyperref}
\usepackage{fancybox}
\usepackage{listings}
\usepackage[abbr]{harvard}

\usepackage[english]{babel}
% or whatever

\usepackage[latin1]{inputenc}
% or whatever

\usepackage{times}
\usepackage[T1]{fontenc}
% Or whatever. Note that the encoding and the font should match. If T1
% does not look nice, try deleting the line with the fontenc.


\usepackage{listings}
 
% "define" Scala
\lstdefinelanguage{scala}{
  morekeywords={abstract,case,catch,class,def,%
    do,else,extends,false,final,finally,%
    for,if,implicit,import,match,mixin,%
    new,null,object,override,package,%
    private,protected,requires,return,sealed,%
    super,this,throw,trait,true,try,%
    type,val,var,while,with,yield},
  otherkeywords={=>,<-,<\%,<:,>:,\#,@},
  sensitive=true,
  morecomment=[l]{//},
  morecomment=[n]{/*}{*/},
  morestring=[b]",
  morestring=[b]',
  morestring=[b]""",
}

\usepackage{color}
\definecolor{dkgreen}{rgb}{0,0.6,0}
\definecolor{gray}{rgb}{0.5,0.5,0.5}
\definecolor{mauve}{rgb}{0.58,0,0.82}
 
% Default settings for code listings
\lstset{frame=tb,
  language=scala,
  aboveskip=2mm,
  belowskip=2mm,
  showstringspaces=false,
  columns=flexible,
  basicstyle={\scriptsize\ttfamily},
  numbers=none,
  numberstyle=\tiny\color{gray},
  keywordstyle=\color{blue},
  commentstyle=\color{dkgreen},
  stringstyle=\color{mauve},
  frame=single,
  breaklines=true,
  breakatwhitespace=true,
  keepspaces=true
  %tabsize=3
}


\title[\course] % (optional, use only with long
                                      % paper titles)
{\lesson}

\subtitle{}
%% {Include Only If Paper Has a Subtitle}

\newcommand{\link}[2]{\href{#1}{\textcolor{blue}{\underline{#2}}}}


% \beamerdefaultoverlayspecification{<+->}


\begin{document}

\begin{frame}
  \titlepage
\end{frame}

%------------------------------------------------------------------------
\begin{frame}[fragile]{Clean Comments}


Comments are (usually) evil.
\begin{itemize}
\item Most comments are compensation for failures to express ideas in code.
\item Comments become baggage when chunkcs of code move.
\item Comments become stale when code changes.
\item Result: comments lie.
\end{itemize}

Comments don't make up for bad code.  If you feel you need a comment to explain some code, put effort into improving the code, not authoring comments for it.


\end{frame}
%------------------------------------------------------------------------

%------------------------------------------------------------------------
\begin{frame}[fragile]{Good Names Can Obviate Comments}


\begin{lstlisting}[language=Java]
// Check to see if the employee is eligible for full benefits
if ((employee.flags & HOURLY_FLAG) && (employee.age > 65))
\end{lstlisting}
We're representing a business rule as a boolean expression and naming it in a comment.  Use the language to express this idea:
\begin{lstlisting}[language=Java]
if (employee.isEligibleForFullBenefits())
\end{lstlisting}
Now if the business rule changes, we know exactly where to change the code that represents it, and the code can be reused.  (What does ``reused'' mean?)\\
Another example:
\begin{lstlisting}[language=Java]
// Returns an instance of the Responder being tested.
protected abstract Responder responderInstance();
\end{lstlisting}
Get rid of the comment and use the language:
\begin{lstlisting}[language=Java]
protected abstract Responder responderBeingTested;
\end{lstlisting}


\end{frame}
%------------------------------------------------------------------------

\section{Good Comments}

%------------------------------------------------------------------------
\begin{frame}[fragile]{Good Comments}


Sometimes comments are useful.
\begin{itemize}
\item Legal comments (copyright notices, licenses)
\item Informative comments
\item Explanation of intent
\item Clarifications
\item Warnings
\item Todos
\item Amplification
\item Javadocs for public APIs
\end{itemize}

\end{frame}
%------------------------------------------------------------------------

%------------------------------------------------------------------------
\begin{frame}[fragile]{Informative Comments}


Sometimes a comment can help the reader of code understand what code is supposed to do even if the code itself has a bug.  Consider:
\begin{lstlisting}[language=Java]
// format matched kk:mm:ss EEE, MMM dd, yyyy Pattern timeMatcher = Pattern.compile(
"\\d*:\\d*:\\d* \\w*, \\w* \\d*, \\d*");
\end{lstlisting}
The regex format is hard to understand, but the comment makes it crystal clear.  If the code has a bug, the programmer knows exactly how to go about fixing it.  (How else might the programmer know?)


\end{frame}
%------------------------------------------------------------------------

%------------------------------------------------------------------------
\begin{frame}[fragile]{Explanation of Intent}


\begin{lstlisting}[language=Java]
public int compareTo(Object o) {
  if(o instanceof WikiPagePath) {
    WikiPagePath p = (WikiPagePath) o;
    String compressedName = StringUtil.join(names, "");
    String compressedArgumentName = StringUtil.join(p.names, "");
    return compressedName.compareTo(compressedArgumentName);
  }
  return 1; // we are greater because we are the right type.
}
\end{lstlisting}
This comment is acceptabe because it explains the programmer's intent: if the type of the other object is different, {\tt this} object is greater

\end{frame}
%------------------------------------------------------------------------

%------------------------------------------------------------------------
\begin{frame}[fragile]{Explanation of Intent}


Consider this test of some multithreaded code:
\begin{lstlisting}[language=Java]
public void testConcurrentAddWidgets() throws Exception {
  // ...

  //This is our best attempt to get a race condition
  //by creating large number of threads.
  for (int i = 0; i < 25000; i++) {
    WidgetBuilderThread widgetBuilderThread =
      new WidgetBuilderThread(widgetBuilder, text, parent, failFlag);
    Thread thread = new Thread(widgetBuilderThread);
    thread.start();
  }
  assertEquals(false, failFlag.get());
}
\end{lstlisting}
The comment quickly explains something that might be puzzling at first glance.

\end{frame}
%------------------------------------------------------------------------

%------------------------------------------------------------------------
\begin{frame}[fragile]{Clarification}


If you're using code from the standard library or other code you can't change to express your idea in code, a clarifying comment can be helpful.
\begin{lstlisting}[language=Java]
public void testCompareTo() throws Exception {
  WikiPagePath a = PathParser.parse("PageA");
  WikiPagePath ab = PathParser.parse("PageA.PageB");
  WikiPagePath b = PathParser.parse("PageB");

  assertTrue(a.compareTo(a) == 0);   // a == a
  assertTrue(a.compareTo(b) != 0);   // a != b
  assertTrue(ab.compareTo(ab) == 0); // ab == ab
  assertTrue(a.compareTo(b) == -1);  // a < b
  // ...
}
\end{lstlisting}

Be very careful though - it's easy to get the comments wrong, which substantially increases the cognitive burden on the reader trying to understand the code.

\end{frame}
%------------------------------------------------------------------------

%------------------------------------------------------------------------
\begin{frame}[fragile]{Warnings}


Consider:
\begin{lstlisting}[language=Java]
public static SimpleDateFormat makeStandardHttpDateFormat() {
  //SimpleDateFormat is not thread safe,
  //so we need to create each instance independently.
  SimpleDateFormat df = new SimpleDateFormat("EEE, dd MMM yyyy HH:mm:ss z");
  df.setTimeZone(TimeZone.getTimeZone("GMT"));
  return df;
}
\end{lstlisting}
On discovering this code you might be tempted to ``optimize'' it by making a single {\tt SimpleDateFormat} to be shared.  The comment would (hopefully) keep you from doing so.

\end{frame}
%------------------------------------------------------------------------

%------------------------------------------------------------------------
\begin{frame}[fragile]{Warnings}


But don't use warning comments when you can express your idea in code.  For example, instead of:
\begin{lstlisting}[language=Java]
// Don't run unless you have some time to kill.
public void _testWithReallyBigFile() {
  writeLinesToFile(10000000);
  response.setBody(testFile);
  response.readyToSend(this);
  String responseString = output.toString();
  assertSubString("Content-Length: 1000000000", responseString); 
  assertTrue(bytesSent > 1000000000);
}
\end{lstlisting}
Use Junit 4's annotations: 
\begin{lstlisting}[language=Java]
@Ignore("Takes too long to run").
public void _testWithReallyBigFile() {
// ...
\end{lstlisting}

\end{frame}
%------------------------------------------------------------------------

\section{Bad Comments}

%------------------------------------------------------------------------
\begin{frame}[fragile]{Bad Comments}


Most comments are bad.  Most bad comments fall into these categories:

\begin{itemize}
\item Redundancies
\item Misleading
\item Noise
\end{itemize}


\end{frame}
%------------------------------------------------------------------------

%------------------------------------------------------------------------
\begin{frame}[fragile]{Redundant and Misleading Comments}


Consider:
\begin{lstlisting}[language=Java]
// Utility method that returns when this.closed is true. 
// Throws an exception if the timeout is reached.
public synchronized void waitForClose(final long timeoutMillis)
        throws Exception {
  if(!closed) {
    wait(timeoutMillis);
    if(!closed)
      throw new Exception("MockResponseSender could not be closed"); }
}
\end{lstlisting}

\begin{itemize}
\item Comment is redundant.  Better to read the code or better yet a well-named method and parameter list.
\item Comment is misleading.  Method doesn't wait for {\tt closed} to become {\tt true} -- it gives it {\tt timeOUtMillis} to become {\tt true} and throws an {\tt Exception} if it doesn't.
\end{itemize}


\end{frame}
%------------------------------------------------------------------------

%------------------------------------------------------------------------
\begin{frame}[fragile]{Noise}

\begin{itemize}
\item Journal comments added each time a file is modified.
\begin{lstlisting}[language=Java]
* Changes (from 11-Oct-2001)
* --------------------------
* 11-Oct-2001 : * Re-organised the class and moved it to new package 
                  com.jrefinery.date (DG;)
* 05-Nov-2001 : * Added a getDescription() method, and eliminated NotableDate 
                  class (DG);
\end{lstlisting}
Don't write journal comments.  Use your VCS.
\item Mandated comments, like Javadocs for code that's not part of a public API.
\item Just plain dumb noise, like:
\begin{lstlisting}[language=Java]
/**
* Default constructor. */
protected AnnualDateRule() { }
\end{lstlisting}

\end{itemize}

\end{frame}
%------------------------------------------------------------------------

%------------------------------------------------------------------------
\begin{frame}[fragile]{Position Markers and Brace Comments}


Sometimes position markers can be useful, but be wary of commenting sections of code like this:
\begin{lstlisting}[language=Java]
// Actions //////////////////////////////////
\end{lstlisting}
And never comment closing braces:
\begin{lstlisting}[language=Java]
    public final E pop() {
        if (isEmpty()) {
            throw new java.util.EmptyStackException();
        } // end if
        return removeNext();
    } // end pop
\end{lstlisting}
Ending-brace comments are well-intentioned but redundant and risk becoming stale when method and class names change.

\end{frame}
%------------------------------------------------------------------------

%------------------------------------------------------------------------
\begin{frame}[fragile]{A few more commenting tips}

\begin{itemize}
\item Don't put attributions in code, like {\tt /* Added by Rick. */}.  Yoru VCS handles this automatically.  (Note, this is different from @author Javadoc tags.)
\item Don't comment out code, delete it.  Again, your VCS will remember old code for you.
\item Don't put HTML markup in comments.  If a tool like Javadoc turns comments into HTML, it's the tool's job to put in the HTML tags, not yours.
\item Javadoc is usually overkill for code that's no part of a public API.
\item Don't put nonlocal information in a comment.  Everything you need to know to understand a comment should be within a couple of lines.
\item Keep comments short.
\end{itemize}

Remember: comments make up for lack of expressivity in a programming language.  You shouldn't need many, and you certainly don't need long comments.


\end{frame}
%------------------------------------------------------------------------


% %------------------------------------------------------------------------
% \begin{frame}[fragile]{}


% \begin{lstlisting}[language=Java]

% \end{lstlisting}

% \begin{itemize}
% \item
% \end{itemize}


% \end{frame}
% %------------------------------------------------------------------------


\end{document}
