\documentclass{beamer}

\newcommand{\course}{CS 2340 Objects and Design}
\newcommand{\lesson}{Clean Boundaries}
\newcommand{\code}{http://www.cc.gatech.edu/~simpkins/teaching/gatech/cs2340/code}

\author[Chris Simpkins] 
{Christopher Simpkins \\\texttt{chris.simpkins@gatech.edu}}
\institute[Georgia Tech] % (optional, but mostly needed)

\date[CS 1331]{}

\subject{\lesson}


% If you have a file called "university-logo-filename.xxx", where xxx
% is a graphic format that can be processed by latex or pdflatex,
% resp., then you can add a logo as follows:

% \pgfdeclareimage[width=0.6in]{coc-logo}{cc_2012_logo}
% \logo{\pgfuseimage{coc-logo}}

\mode<presentation>
{
  \usetheme{Berlin}
  \useoutertheme{infolines}

  % or ...

 \setbeamercovered{transparent}
  % or whatever (possibly just delete it)
}

\usepackage{hyperref}
\usepackage{fancybox}
\usepackage{listings}
\usepackage[abbr]{harvard}

\usepackage[english]{babel}
% or whatever

\usepackage[latin1]{inputenc}
% or whatever

\usepackage{times}
\usepackage[T1]{fontenc}
% Or whatever. Note that the encoding and the font should match. If T1
% does not look nice, try deleting the line with the fontenc.


\usepackage{listings}
 
% "define" Scala
\lstdefinelanguage{scala}{
  morekeywords={abstract,case,catch,class,def,%
    do,else,extends,false,final,finally,%
    for,if,implicit,import,match,mixin,%
    new,null,object,override,package,%
    private,protected,requires,return,sealed,%
    super,this,throw,trait,true,try,%
    type,val,var,while,with,yield},
  otherkeywords={=>,<-,<\%,<:,>:,\#,@},
  sensitive=true,
  morecomment=[l]{//},
  morecomment=[n]{/*}{*/},
  morestring=[b]",
  morestring=[b]',
  morestring=[b]""",
}

\usepackage{color}
\definecolor{dkgreen}{rgb}{0,0.6,0}
\definecolor{gray}{rgb}{0.5,0.5,0.5}
\definecolor{mauve}{rgb}{0.58,0,0.82}
 
% Default settings for code listings
\lstset{frame=tb,
  language=scala,
  aboveskip=2mm,
  belowskip=2mm,
  showstringspaces=false,
  columns=flexible,
  basicstyle={\scriptsize\ttfamily},
  numbers=none,
  numberstyle=\tiny\color{gray},
  keywordstyle=\color{blue},
  commentstyle=\color{dkgreen},
  stringstyle=\color{mauve},
  frame=single,
  breaklines=true,
  breakatwhitespace=true,
  keepspaces=true
  %tabsize=3
}


\title[\course] % (optional, use only with long
                                      % paper titles)
{\lesson}

\subtitle{}
%% {Include Only If Paper Has a Subtitle}

\newcommand{\link}[2]{\href{#1}{\textcolor{blue}{\underline{#2}}}}


% \beamerdefaultoverlayspecification{<+->}


\begin{document}

\begin{frame}
  \titlepage
\end{frame}

\section{Using Third Party Code}

%------------------------------------------------------------------------
\begin{frame}[fragile]{Library Providers versus Library Users}


Tension between providers of an interface and users of an interface
\begin{itemize}
\item Providers want to produce maximally flexible and useful functionality for a wide audience.
\item Users want to focus on their needs.
\end{itemize}

This tension is particularly acute in standard library classes.  Consider {\tt java.util.Map} ...

\end{frame}
%------------------------------------------------------------------------

%------------------------------------------------------------------------
\begin{frame}[fragile]{A Key-Value Sensors Data Structure}


Say you have an application that needs to maintain sensors that are identified by ids.  This is precisely what {\tt Map}s are for, so you use a {\tt Map}.  But ...

\begin{itemize}
\item your code passes around sensors, and most parts of your app should only retrieve sensors from the {\tt Map}, and
\item {\tt java.util.Map} includes methods that allow users to delete items from the map.
\end{itemize}

The provider of {\tt Map} is providing flexibility we don't want.

\end{frame}
%------------------------------------------------------------------------

%------------------------------------------------------------------------
\begin{frame}[fragile]{How to Deal with Overly General Library Classes}


If we use {\tt Map} to hold sensors, we end up with code like this:
\begin{lstlisting}[language=Java]
Map<Sensor> sensors = new HashMap<Sensor>();
...
Sensor s = sensors.get(sensorId );
\end{lstlisting}

There's a concept here that's begging to be represented in our system:
\begin{lstlisting}[language=Java]
public class Sensors {
  private Map<String, Sensor> sensors = new HashMap<>();

  public Sensor getById(String id) {
    return sensors.get(id);
  }
  // ...
}
\end{lstlisting}

\begin{itemize}
\item Wrapping not necessary for every use of {\tt Map}, but generally not a good idea to pass {\tt Map}s around.
\item This ``wrapping'' technique is a good general approach to deal with overly general third-party libraries at boundaries.
\end{itemize}

\end{frame}
%------------------------------------------------------------------------

%------------------------------------------------------------------------
\begin{frame}[fragile]{Exploring and Learning Boundaries}


When you need a capability, you can write the code yourself, or use a library that provides the capability.
\begin{itemize}
\item If a libary is available, ``buy, don't build.''
\item Still have to learn the library.
\end{itemize}

Instead of studying documentation, write {\it learning tests}.


\end{frame}
%------------------------------------------------------------------------

%------------------------------------------------------------------------
\begin{frame}[fragile]{Learning {\tt log4j}}


We write a simple test the way we think the library should work:
\begin{lstlisting}[language=Java]
@Test
public void testLogCreate() {
  Logger logger = Logger.getLogger("MyLogger");
  logger.info("hello");
}
\end{lstlisting}

When the test runs, we get an error saying that we need an {\tt Appender}, so we look up {\tt Appender}s in thr docs and update our test:
\begin{lstlisting}[language=Java]
@Test
public void testLogAddAppender() {
  Logger logger = Logger.getLogger("MyLogger");
  ConsoleAppender appender = new ConsoleAppender();
  logger.addAppender(appender);
  logger.info("hello");
}
\end{lstlisting}

This time we get an error about a missing output stream, ...


\end{frame}
%------------------------------------------------------------------------

%------------------------------------------------------------------------
\begin{frame}[fragile]{Learning Tests Lead to Boundary Tests}

... so we read a little more and come up with:
\begin{lstlisting}[language=Java]
@Test
public void testLogAddAppender() {
  Logger logger = Logger.getLogger("MyLogger");
  logger.removeAllAppenders();
  logger.addAppender(new ConsoleAppender(
      new PatternLayout("%p %t %m%n"),
      ConsoleAppender.SYSTEM_OUT));
  logger.info("hello");
}
\end{lstlisting}

After a few more iterations of writing tests and reading docs, we know how {\tt log4j} works and we have a set of tests with example code.
\begin{itemize}
\item These tests are {\it boundary tests} that use the library the same way as our production code.
\item When a new version of the library is released, we can test it with our boundary tests before integrating it with production code.
\end{itemize}


\end{frame}
%------------------------------------------------------------------------

%------------------------------------------------------------------------
\begin{frame}[fragile]{The Adapter Pattern\footnote{GoF, {\it Design Patterns}, 1994} (A.K.A Wrapper)}



{\bf Intent}: Convert the interface of a class into another interface clients expect. Adapter lets classes work together that couldn't otherwise because of incompatible interfaces.

{\bf Structure}
\vspace{-.05in}
\begin{center}
\includegraphics[height=1.2in]{adapter-pattern.png}\\
\end{center}
\vspace{-.125in}
{\bf Participants}
\begin{itemize}
\item {\bf Target} defines the domain-specific interface that Client uses.
\item {\bf Client} collaborates with objects conforming to the Target interface.
\item {\bf Adaptee} defines an existing interface that needs adapting.
\item {\bf Adapter} adapts the interface of Adaptec to the Target interface.
\end{itemize}


\end{frame}
%------------------------------------------------------------------------


%------------------------------------------------------------------------
\begin{frame}[fragile]{Integrating Future Code}

\begin{itemize}
\item Imagine we're a team writing an application that will use a hardware transmitter, but the transmitter's software is handled by another team that hasn't defined their software interface.
\item We can define our own interface the way {\it we} want it to work.
\item While we're waiting for the transmitter team, we create a fake implementation to work with.
\item When the transmitter team finally gives us their interface, we can write an adapter to fit it to our interface.
\item The rest of our code is unaffected.
\end{itemize}

\begin{center}
\includegraphics[height=1.25in]{transmitter-adapter.png}
\end{center}


\end{frame}
%------------------------------------------------------------------------

%------------------------------------------------------------------------
\begin{frame}[fragile]{Final Thoughts on Boundaries}


\begin{itemize}
\item Third-party libraries out of our control.
\item Boundary tests minimize surprises when migrating to new versions of libraries.
\item Minimize the number of places in our code that accesses third-party libraries.
\item Sequester third-party libraries in boundary code using the adapter pattern.
\end{itemize}


\end{frame}
%------------------------------------------------------------------------

% %------------------------------------------------------------------------
% \begin{frame}[fragile]{}


% \begin{lstlisting}[language=Java]

% \end{lstlisting}

% \begin{itemize}
% \item
% \end{itemize}


% \end{frame}
% %------------------------------------------------------------------------


\end{document}
