\documentclass{beamer}

\newcommand{\course}{CS 2340 Objects and Design}
\newcommand{\lesson}{Clean Formatting}
\newcommand{\code}{http://www.cc.gatech.edu/~simpkins/teaching/gatech/cs2340/code}

\author[Chris Simpkins] 
{Christopher Simpkins \\\texttt{chris.simpkins@gatech.edu}}
\institute[Georgia Tech] % (optional, but mostly needed)

\date[CS 1331]{}

\subject{\lesson}


% If you have a file called "university-logo-filename.xxx", where xxx
% is a graphic format that can be processed by latex or pdflatex,
% resp., then you can add a logo as follows:

% \pgfdeclareimage[width=0.6in]{coc-logo}{cc_2012_logo}
% \logo{\pgfuseimage{coc-logo}}

\mode<presentation>
{
  \usetheme{Berlin}
  \useoutertheme{infolines}

  % or ...

 \setbeamercovered{transparent}
  % or whatever (possibly just delete it)
}

\usepackage{hyperref}
\usepackage{fancybox}
\usepackage{listings}
\usepackage[abbr]{harvard}

\usepackage[english]{babel}
% or whatever

\usepackage[latin1]{inputenc}
% or whatever

\usepackage{times}
\usepackage[T1]{fontenc}
% Or whatever. Note that the encoding and the font should match. If T1
% does not look nice, try deleting the line with the fontenc.


\usepackage{listings}
 
% "define" Scala
\lstdefinelanguage{scala}{
  morekeywords={abstract,case,catch,class,def,%
    do,else,extends,false,final,finally,%
    for,if,implicit,import,match,mixin,%
    new,null,object,override,package,%
    private,protected,requires,return,sealed,%
    super,this,throw,trait,true,try,%
    type,val,var,while,with,yield},
  otherkeywords={=>,<-,<\%,<:,>:,\#,@},
  sensitive=true,
  morecomment=[l]{//},
  morecomment=[n]{/*}{*/},
  morestring=[b]",
  morestring=[b]',
  morestring=[b]""",
}

\usepackage{color}
\definecolor{dkgreen}{rgb}{0,0.6,0}
\definecolor{gray}{rgb}{0.5,0.5,0.5}
\definecolor{mauve}{rgb}{0.58,0,0.82}
 
% Default settings for code listings
\lstset{frame=tb,
  language=scala,
  aboveskip=2mm,
  belowskip=2mm,
  showstringspaces=false,
  columns=flexible,
  basicstyle={\scriptsize\ttfamily},
  numbers=none,
  numberstyle=\tiny\color{gray},
  keywordstyle=\color{blue},
  commentstyle=\color{dkgreen},
  stringstyle=\color{mauve},
  frame=single,
  breaklines=true,
  breakatwhitespace=true,
  keepspaces=true
  %tabsize=3
}


\title[\course] % (optional, use only with long
                                      % paper titles)
{\lesson}

\subtitle{}
%% {Include Only If Paper Has a Subtitle}

\newcommand{\link}[2]{\href{#1}{\textcolor{blue}{\underline{#2}}}}


% \beamerdefaultoverlayspecification{<+->}


\begin{document}

\begin{frame}
  \titlepage
\end{frame}

%------------------------------------------------------------------------
\begin{frame}[fragile]{Formatting}

\begin{quote}
Code should be written for human beings to understand, and only incidentally for machines to execute. -- Hal Abelson and Gerald Sussman, SICP
\end{quote}

\begin{quote}
The purpose of a computer program is to tell other people what you want the computer to do. -- Donald Knuth
\end{quote}


The purpose of formatting is to facilitate communication.  The formatting of code conveys information to the reader.



\end{frame}
%------------------------------------------------------------------------

%------------------------------------------------------------------------
\begin{frame}[fragile]{Vertical Formatting}

\begin{itemize}
\item Newspaper metaphor
\item Vertical openness between concepts
\item Vertical density
\item Vertical distance
\item Vertical ordering
\end{itemize}


\end{frame}
%------------------------------------------------------------------------

%------------------------------------------------------------------------
\begin{frame}[fragile]{Vertical Openness Between Concepts}


Notice how vertical openness helps us locate concepts in the code more quickly.
\begin{lstlisting}[language=Java]
package fitnesse.wikitext.widgets;

import java.util.regex.*;

public class BoldWidget extends ParentWidget {
  public static final String REGEXP = "'''.+?'''";
  private static final Pattern pattern = Pattern.compile("'''(.+?)'''",
    Pattern.MULTILINE + Pattern.DOTALL
  );

  public BoldWidget(ParentWidget parent, String text) throws Exception { 
    super(parent);
    Matcher match = pattern.matcher(text);
    match.find();
    addChildWidgets(match.group(1));
  }
}
\end{lstlisting}

\end{frame}
%------------------------------------------------------------------------

%------------------------------------------------------------------------
\begin{frame}[fragile]{Vertical Openness Between Concepts}


If we leave out the blank lines:
\vspace{-.05in}
\begin{lstlisting}[language=Java]
package fitnesse.wikitext.widgets;
import java.util.regex.*;
public class BoldWidget extends ParentWidget {
  public static final String REGEXP = "'''.+?'''";
  private static final Pattern pattern = Pattern.compile("'''(.+?)'''",
    Pattern.MULTILINE + Pattern.DOTALL
  );
  public BoldWidget(ParentWidget parent, String text) throws Exception { 
    super(parent);
    Matcher match = pattern.matcher(text);
    match.find();
    addChildWidgets(match.group(1));
  }
}
\end{lstlisting}
\vspace{-.05in}
\begin{itemize}
\item It's harder to distinguish the package statement, the beginning and end of the imports, and the class declaration.
\item It's harder to locate where the instance variables end and methods begin.
\end{itemize}

\end{frame}
%------------------------------------------------------------------------

%------------------------------------------------------------------------
\begin{frame}[fragile]{Vertical Density}

\vspace{-.05in}
Openness separates concepts. Density implies association. Consider:
\vspace{-.25in}
\begin{lstlisting}[language=Java]
public class ReporterConfig {
  /** The class name of the reporter listener */
  private String m_className;

  /** The properties of the reporter listener */
  private List<Property> m_properties = new ArrayList<Property>();

  public void addProperty(Property property) {
    m_properties.add(property);
  }
\end{lstlisting}
\vspace{-.05in}
The vertical openness (and bad comments) misleads the reader.  Better to use closeness to convey relatedness:
\vspace{-.05in}
\begin{lstlisting}[language=Java]
public class ReporterConfig {
  private String m_className;
  private List<Property> m_properties = new ArrayList<Property>();

  public void addProperty(Property property) {
    m_properties.add(property);
  }
}
\end{lstlisting}

\end{frame}
%------------------------------------------------------------------------

%------------------------------------------------------------------------
\begin{frame}[fragile]{Vertical Distance and Ordering}


Concepts that are closely related should be vertically close to each other.
\begin{itemize}
\item Variables should be declared as close to their usage as possible.
\item Instance variables should be declared at the top of the class.
\item Dependent functions: callers should be above callees.
\end{itemize}


\end{frame}
%------------------------------------------------------------------------

%------------------------------------------------------------------------
\begin{frame}[fragile]{Horizontal Openness and Density}

\begin{itemize}
\item Keep lines short.  Uncle Bob says 120, but he's wrong.  Keep your lines at 80 characters or fewer if possible (sometimes it is impossible, but very rarely).
\item Put spaces around {\tt =} to accentuate the distinction between the LHS and RHS.
\item Don't put spaces between method names and parens, or parens and paramter lists - they're closely related, so should be close.
\item Use spaces to accentuate operator precedence, e.g., no space between unary operators and their operands, space between binary operators and their operands.
\item Don't try to horizontally align lists of assignments -- it draws attention to the wrong thing and can be misleading, e.g., encouraging the reader to read down a column.
\item Always indent scopes (classes, methods, blocks).
\end{itemize}


\end{frame}
%------------------------------------------------------------------------

%------------------------------------------------------------------------
\begin{frame}[fragile]{Team Rules}

\begin{itemize}
\item Every team should agree on a coding standard and everyone should adhere to it.
\item Don't modify a file just to change the formatting, but if you are modifying it anyway, go ahead and fix the formatting of the code you modify.
\item Code formatting standards get religious.  My rule: make your code look like the language inventor's code.
\item If the language you're using has a code convention (like Java's), use it!
\end{itemize}


\end{frame}
%------------------------------------------------------------------------


% %------------------------------------------------------------------------
% \begin{frame}[fragile]{}


% \begin{lstlisting}[language=Java]

% \end{lstlisting}

% \begin{itemize}
% \item
% \end{itemize}


% \end{frame}
% %------------------------------------------------------------------------


\end{document}
