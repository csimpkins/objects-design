\documentclass{beamer}

\newcommand{\course}{CS 2340 Objects and Design}
\newcommand{\lesson}{Creational Patterns}
\newcommand{\code}{http://www.cc.gatech.edu/~simpkins/teaching/gatech/cs2340/code}

\author[Chris Simpkins] 
{Christopher Simpkins \\\texttt{chris.simpkins@gatech.edu}}
\institute[Georgia Tech] % (optional, but mostly needed)

\date[CS 1331]{}

\subject{\lesson}


% If you have a file called "university-logo-filename.xxx", where xxx
% is a graphic format that can be processed by latex or pdflatex,
% resp., then you can add a logo as follows:

% \pgfdeclareimage[width=0.6in]{coc-logo}{cc_2012_logo}
% \logo{\pgfuseimage{coc-logo}}

\mode<presentation>
{
  \usetheme{Berlin}
  \useoutertheme{infolines}

  % or ...

 \setbeamercovered{transparent}
  % or whatever (possibly just delete it)
}

\usepackage{hyperref}
\usepackage{fancybox}
\usepackage{listings}
\usepackage[abbr]{harvard}

\usepackage[english]{babel}
% or whatever

\usepackage[latin1]{inputenc}
% or whatever

\usepackage{times}
\usepackage[T1]{fontenc}
% Or whatever. Note that the encoding and the font should match. If T1
% does not look nice, try deleting the line with the fontenc.


\usepackage{listings}
 
% "define" Scala
\lstdefinelanguage{scala}{
  morekeywords={abstract,case,catch,class,def,%
    do,else,extends,false,final,finally,%
    for,if,implicit,import,match,mixin,%
    new,null,object,override,package,%
    private,protected,requires,return,sealed,%
    super,this,throw,trait,true,try,%
    type,val,var,while,with,yield},
  otherkeywords={=>,<-,<\%,<:,>:,\#,@},
  sensitive=true,
  morecomment=[l]{//},
  morecomment=[n]{/*}{*/},
  morestring=[b]",
  morestring=[b]',
  morestring=[b]""",
}

\usepackage{color}
\definecolor{dkgreen}{rgb}{0,0.6,0}
\definecolor{gray}{rgb}{0.5,0.5,0.5}
\definecolor{mauve}{rgb}{0.58,0,0.82}
 
% Default settings for code listings
\lstset{frame=tb,
  language=scala,
  aboveskip=2mm,
  belowskip=2mm,
  showstringspaces=false,
  columns=flexible,
  basicstyle={\scriptsize\ttfamily},
  numbers=none,
  numberstyle=\tiny\color{gray},
  keywordstyle=\color{blue},
  commentstyle=\color{dkgreen},
  stringstyle=\color{mauve},
  frame=single,
  breaklines=true,
  breakatwhitespace=true,
  keepspaces=true
  %tabsize=3
}


\title[\course] % (optional, use only with long
                                      % paper titles)
{\lesson}

\subtitle{}
%% {Include Only If Paper Has a Subtitle}

\newcommand{\link}[2]{\href{#1}{\textcolor{blue}{\underline{#2}}}}


% \beamerdefaultoverlayspecification{<+->}


\begin{document}

\begin{frame}
  \titlepage
\end{frame}

%------------------------------------------------------------------------
\begin{frame}[fragile]{Creational Design Patterns}


Abstracts the instantiation process.
\begin{itemize}
\item Encapsulate knowledge about which concrete classes the system uses.
\item Hide how instances of these classes are created and put together.
\end{itemize}


\end{frame}
%------------------------------------------------------------------------

%------------------------------------------------------------------------
\begin{frame}[fragile]{Abstract Factory}


{\bf Intent}: Provide an interface for creating families of related or dependent objects without specifying their concrete classes.

{\bf Structure}
\vspace{-.15in}
\begin{center}
\includegraphics[height=1.2in]{abstract-factory-pattern.png}\\
\end{center}
\vspace{-.225in}
{\bf Participants}
\begin{itemize}
\item {\bf AbstractFactory} declares an interface for operations that create abstract product objects.
\item {\bf ConcreteFactory} implements the operations to create concrete product objects.
\item {\bf AbstractProduct} declares an interface for a type of product.
\item {\bf ConcreteProduct} defines a product object to be created by the corresponding concrete factory; implements the AbstractProduct interface.
\item {\bf Client} uses only interfaces declared by AbstractFactory and AbstractProduct classes.
\end{itemize}


\end{frame}
%------------------------------------------------------------------------

%------------------------------------------------------------------------
\begin{frame}[fragile]{Factory Method (a.k.a. Virtual Constructor)}

\vspace{-.05in}
{\bf Intent}: Define an interface for creating an object, but let subclasses decide which class to instantiate. Factory Method lets a class defer instantiation to subclasses.

{\bf Structure}
\vspace{-.1in}
\begin{center}
\includegraphics[height=1.1in]{factory-method-pattern.png}\\
\end{center}
\vspace{-.2in}
{\bf Participants}
\begin{itemize}
\item {\bf Product} defines the interface of objects the factory creates.
\item {\bf ConcreteProduct} implements the Product interface.
\item {\bf Creator} declares the factory method, which returns an object of type Product.
\item {\bf ConcreteCreator} overrides factory method to return a ConcreteProduct object.
\end{itemize}


\end{frame}
%------------------------------------------------------------------------

%------------------------------------------------------------------------
\begin{frame}[fragile]{Factory Method Example: Active Records (1 of 4)}


Say we have a solution domain object that represents a problem domain entity:
\begin{lstlisting}[language=Java]
public class Person {

    protected final int id;
    protected String name;
   
    public Person(int id, String name) {
        this.id = id;
        this.name = name;
    }

    public int getId() { return id; }
    public String getName() { return name; }
    public void setName(String name) { this.name = name; }
}
\end{lstlisting}
How can we add persistence capability in an abstract way so that we can swap out different persistence implementations (database, etc.)?

\end{frame}
%------------------------------------------------------------------------

%------------------------------------------------------------------------
\begin{frame}[fragile]{Factory Method Example: Active Records (2 of 4)}


Active Records are objects that  know how to store and retrieve themselves from a data store.  The simplest implementation of an ActiveRecord uses an abstract class:
\begin{lstlisting}[language=Java]
public abstract class ActivePerson extends Person {

    public ActivePerson(int id, String name) {
        super(id, name);
    }

    public abstract Person createNew(String name);

    public abstract Person findById(int id);

    public abstract void save();
}
\end{lstlisting}
\vspace{-.05in}
{\tt ActivePerson} extends {\tt Person} with persistence capabilities.  Now applications that use a particular data store can sublcass {\tt ActivePerson} and implement data store-specific versions of these persistence methods.


\end{frame}
%------------------------------------------------------------------------

%------------------------------------------------------------------------
\begin{frame}[fragile]{Factory Method Example: Active Records (3 of 4)}


Here's a subclass of {\tt ActivePerson} that uses a {\tt HashMap}:
\vspace{-.05in}
\begin{lstlisting}[language=Java]
public class HashMapPerson extends ActivePerson {

    private static HashMap<Integer, Person> persons = new HashMap<>();
    private static int lastUsedId = 0;

    protected HashMapPerson(int id, String name) {
        super(id, name);
    }

    public Person createNew(String name) {
        Person newPerson = new HashMapPerson(lastUsedId++, name);
        persons.put(newId, newPerson);
        return newPerson;
    }
    public Person findById(int id) {
        return persons.get(id);
    }
    public void save() {
      // nothing to do - client has alias to object in HashMap
    }    
}
\end{lstlisting}

\end{frame}
%------------------------------------------------------------------------

%------------------------------------------------------------------------
\begin{frame}[fragile]{Factory Method Example: Active Records (4 of 4)}


Benefits of using {\tt ActivePerson}:
\begin{itemize}
\item A {\tt MySQLPerson} would implement MySQL-specific code that maps relational database reperesentations of objects to their Java object counterparts.
\item Application is coded to {\tt ActivePerson} interface so versions of {\tt ActivePerson} that use different data stores can be swapped out by changing only the client code that instantiates the {\tt ActivePerson} objects.
\item You could put all of your active record-instantiating code in an Abstract Factory or a registry (which could be a singleton) so there's only one place to make this change for all kinds of peristed objects.
\end{itemize}

There are other ways of doing this, but active records are easy to understand.  All object-relational mapping and data store frameworks use these concepts.

\end{frame}
%------------------------------------------------------------------------



%------------------------------------------------------------------------
\begin{frame}[fragile]{Singleton}


{\bf Intent}: Ensure a class only has one instance, and provide a global point of access to it.

{\bf Structure}
\vspace{-.15in}
\begin{center}
\includegraphics[height=1in]{singleton-pattern.png}\\
\end{center}
\vspace{-.15in}
{\bf Participants}
\begin{itemize}
\item {\bf Singleton} defines an Instance operation that lets clients access its unique instance.
\begin{itemize}
\item Instance is a class operation (that is, a class method in Smalltalk and a static member function in C++, or static method in Java).
\item May be responsible for creating its own unique instance.
\end{itemize}
\end{itemize}


\end{frame}
%------------------------------------------------------------------------

% %------------------------------------------------------------------------
% \begin{frame}[fragile]{}


% \begin{lstlisting}[language=Java]

% \end{lstlisting}

% \begin{itemize}
% \item
% \end{itemize}


% \end{frame}
% %------------------------------------------------------------------------


\end{document}
