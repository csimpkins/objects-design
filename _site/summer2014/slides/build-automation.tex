\documentclass{beamer}

\newcommand{\course}{CS 2340 Objects and Design}
\newcommand{\lesson}{Build Automation}
\newcommand{\code}{http://www.cc.gatech.edu/~simpkins/teaching/gatech/cs2340/code}

\author[Chris Simpkins]
{Christopher Simpkins \\\texttt{chris.simpkins@gatech.edu}}
\institute[Georgia Tech] % (optional, but mostly needed)

\date[CS 1331]{}

\subject{\lesson}


% If you have a file called "university-logo-filename.xxx", where xxx
% is a graphic format that can be processed by latex or pdflatex,
% resp., then you can add a logo as follows:

% \pgfdeclareimage[width=0.6in]{coc-logo}{cc_2012_logo}
% \logo{\pgfuseimage{coc-logo}}

\mode<presentation>
{
  \usetheme{Berlin}
  \useoutertheme{infolines}

  % or ...

 \setbeamercovered{transparent}
  % or whatever (possibly just delete it)
}

\usepackage{hyperref}
\usepackage{fancybox}
\usepackage{listings}
\usepackage[abbr]{harvard}

\usepackage[english]{babel}
% or whatever

\usepackage[latin1]{inputenc}
% or whatever

\usepackage{times}
\usepackage[T1]{fontenc}
% Or whatever. Note that the encoding and the font should match. If T1
% does not look nice, try deleting the line with the fontenc.


\usepackage{listings}
 
% "define" Scala
\lstdefinelanguage{scala}{
  morekeywords={abstract,case,catch,class,def,%
    do,else,extends,false,final,finally,%
    for,if,implicit,import,match,mixin,%
    new,null,object,override,package,%
    private,protected,requires,return,sealed,%
    super,this,throw,trait,true,try,%
    type,val,var,while,with,yield},
  otherkeywords={=>,<-,<\%,<:,>:,\#,@},
  sensitive=true,
  morecomment=[l]{//},
  morecomment=[n]{/*}{*/},
  morestring=[b]",
  morestring=[b]',
  morestring=[b]""",
}

\usepackage{color}
\definecolor{dkgreen}{rgb}{0,0.6,0}
\definecolor{gray}{rgb}{0.5,0.5,0.5}
\definecolor{mauve}{rgb}{0.58,0,0.82}
 
% Default settings for code listings
\lstset{frame=tb,
  language=scala,
  aboveskip=2mm,
  belowskip=2mm,
  showstringspaces=false,
  columns=flexible,
  basicstyle={\scriptsize\ttfamily},
  numbers=none,
  numberstyle=\tiny\color{gray},
  keywordstyle=\color{blue},
  commentstyle=\color{dkgreen},
  stringstyle=\color{mauve},
  frame=single,
  breaklines=true,
  breakatwhitespace=true,
  keepspaces=true
  %tabsize=3
}


\title[\course] % (optional, use only with long
                                      % paper titles)
{\lesson}

\subtitle{}
%% {Include Only If Paper Has a Subtitle}

\newcommand{\link}[2]{\href{#1}{\textcolor{blue}{\underline{#2}}}}


% \beamerdefaultoverlayspecification{<+->}


\begin{document}

\begin{frame}
  \titlepage
\end{frame}


%------------------------------------------------------------------------
\begin{frame}[fragile]{Ant}

\vspace{-.07in}
\link{http://ant.apache.org/}{Ant} is a build automation tool, like {\tt make}.  Install it however you want (like with homebrew on a Mac), then put this in a file named {\tt build.xml} in the root directory of your blackjack project:
\vspace{-.07in}
\begin{lstlisting}[language=xml]
<?xml version="1.0" encoding="UTF-8"?>
<project name="blackjack" default="default" basedir=".">
  <path id="classpath">
    <fileset dir="target/classes" includes="**/*.class"/>
  </path>

  <target name="init">
    <mkdir dir="target"/>
    <mkdir dir="target/classes"/>
  </target>

  <target name="compile" depends="init">
    <javac srcdir="src/main/java"
           destdir="target/classes"
           classpathref="classpath"
           source="1.7"
           target="1.7" />
  </target>
</project>
\end{lstlisting}

\begin{itemize}
\item
\end{itemize}


\end{frame}
%------------------------------------------------------------------------

%------------------------------------------------------------------------
\begin{frame}[fragile]{Compiling with Ant}


Invoke the {\tt compile} target to compile the project:
\begin{lstlisting}[language=Java]
$ ant compile
Buildfile: /Users/chris/scratch/blackjack/build.xml

init:
...
compile:
...
BUILD SUCCESSFUL
Total time: 0 seconds
\end{lstlisting}

This will produce class files in {\tt target/classes}.  How would you run the {\tt Blackjack} class?

\end{frame}
%------------------------------------------------------------------------


%------------------------------------------------------------------------
\begin{frame}[fragile]{Runnable Jar Files}


The hard way: \link{http://docs.oracle.com/javase/tutorial/deployment/jar/appman.html}{Oracle's Jar docs}\\
\vspace{.1in}
The easy way: add this to your {\tt build.xml}:
\begin{lstlisting}[language=xml]
  <target name="package" depends="compile">
    <jar destfile="target/blackjack.jar">
      <fileset dir="target/classes"/>
      <manifest>
        <attribute name="Main-class"
                   value="edu.gatech.cs2340.blackjack.Blackjack"/>
      </manifest>
    </jar>
  </target>
\end{lstlisting}

Then you can do:

\begin{lstlisting}[language=bash]
$ ant package
...
BUILD SUCCESSFUL
Total time: 0 seconds
$ java -jar target/blackjack.jar
\end{lstlisting}



\end{frame}
%------------------------------------------------------------------------


%------------------------------------------------------------------------
\begin{frame}[fragile]{Generating Documentation}

Add this target to your {\tt build.xml}
\begin{lstlisting}[language=Java]
<target name="javadoc" description="Generate Javadoc" depends="compile">
    <javadoc destdir="target/docs/api"
             classpathref="classpath"
             access="private"
             version="true"
             use="true"
             author="true"
             overview="src/main/java/overview.html">
      <fileset dir="src/main/java" defaultexcludes="yes">
        <include name="**/*.java" />
      </fileset>
      <doctitle>
        <![CDATA[<h1>Blackjack API Documentation</h1>]]>
      </doctitle>
      <bottom>
        <![CDATA[<i>Copyright &#169; Georgia Tech. All Rights Reserved.</i>]]>
      </bottom>

      <link href="http://docs.oracle.com/javase/7/docs/api/"/>
    </javadoc>
  </target>
\end{lstlisting}



\end{frame}
%------------------------------------------------------------------------

% %------------------------------------------------------------------------
% \begin{frame}[fragile]{}

% \begin{itemize}
% \item
% \end{itemize}

% \begin{lstlisting}[language=Java]

% \end{lstlisting}



% \end{frame}
% %------------------------------------------------------------------------


\end{document}
