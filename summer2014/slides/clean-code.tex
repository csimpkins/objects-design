\documentclass{beamer}

\newcommand{\course}{CS 2340 Objects and Design}
\newcommand{\lesson}{Clean Code}
\newcommand{\code}{http://www.cc.gatech.edu/~simpkins/teaching/gatech/cs2340/code}

\author[Chris Simpkins] {}
\institute[Georgia Tech] % (optional, but mostly needed)

\date[CS 1331]{}

\subject{\lesson}


% If you have a file called "university-logo-filename.xxx", where xxx
% is a graphic format that can be processed by latex or pdflatex,
% resp., then you can add a logo as follows:

% \pgfdeclareimage[width=0.6in]{coc-logo}{cc_2012_logo}
% \logo{\pgfuseimage{coc-logo}}

\mode<presentation>
{
  \usetheme{Berlin}
  \useoutertheme{infolines}

  % or ...

 \setbeamercovered{transparent}
  % or whatever (possibly just delete it)
}

\usepackage{hyperref}
\usepackage{fancybox}
\usepackage{listings}
\usepackage[abbr]{harvard}

\usepackage[english]{babel}
% or whatever

\usepackage[latin1]{inputenc}
% or whatever

\usepackage{times}
\usepackage[T1]{fontenc}
% Or whatever. Note that the encoding and the font should match. If T1
% does not look nice, try deleting the line with the fontenc.


\usepackage{listings}
 
% "define" Scala
\lstdefinelanguage{scala}{
  morekeywords={abstract,case,catch,class,def,%
    do,else,extends,false,final,finally,%
    for,if,implicit,import,match,mixin,%
    new,null,object,override,package,%
    private,protected,requires,return,sealed,%
    super,this,throw,trait,true,try,%
    type,val,var,while,with,yield},
  otherkeywords={=>,<-,<\%,<:,>:,\#,@},
  sensitive=true,
  morecomment=[l]{//},
  morecomment=[n]{/*}{*/},
  morestring=[b]",
  morestring=[b]',
  morestring=[b]""",
}

\usepackage{color}
\definecolor{dkgreen}{rgb}{0,0.6,0}
\definecolor{gray}{rgb}{0.5,0.5,0.5}
\definecolor{mauve}{rgb}{0.58,0,0.82}
 
% Default settings for code listings
\lstset{frame=tb,
  language=scala,
  aboveskip=2mm,
  belowskip=2mm,
  showstringspaces=false,
  columns=flexible,
  basicstyle={\scriptsize\ttfamily},
  numbers=none,
  numberstyle=\tiny\color{gray},
  keywordstyle=\color{blue},
  commentstyle=\color{dkgreen},
  stringstyle=\color{mauve},
  frame=single,
  breaklines=true,
  breakatwhitespace=true,
  keepspaces=true
  %tabsize=3
}


\title[\course] % (optional, use only with long
                                      % paper titles)
{\lesson}

\subtitle{}
%% {Include Only If Paper Has a Subtitle}

\newcommand{\link}[2]{\href{#1}{\textcolor{blue}{\underline{#2}}}}


% \beamerdefaultoverlayspecification{<+->}


\begin{document}

\begin{frame}
  \titlepage
\vspace{-1.1in}
\begin{center}
\includegraphics[height=2.5in]{clean-code-book.png}
\end{center}
\end{frame}

%------------------------------------------------------------------------
\begin{frame}[fragile]{Clean Code}


What is ``clean code?''
\begin{itemize}
\item Elegant and efficient. -- Bjarne Stroustrup
\item Simple and direct. Readable.  -- Grady Booch
\item Understandable by others, tested, literate. -- Dave Thomas
\item Code works pretty much as expected.  Beatuful code looks like the language was made for the problem. -- Ward Cunningham
\end{itemize}

Why do we care abou clean code?
\begin{itemize}
\item Messes are costly.  Quick and dirty to get it done ends up not getting it done and you will not enjoy it.  It's lose-lose!
\item We are professionals who care about our craft. 
\end{itemize}

The Boy Scout Rule

\end{frame}
%------------------------------------------------------------------------

\section{Clean Names}

%------------------------------------------------------------------------
\begin{frame}[fragile]{Meaningful Names}


\begin{itemize}
\item The name of a variable, method, or class should reveal its purpose.
\item If you feel the need to comment on the name itself, pick a better name.
\item Code with a dictionary close at hand.
\end{itemize}

Don't ever do this!
\begin{lstlisting}[language=Java]
int d; // elapsed time in days
\end{lstlisting}

Much better:
\begin{lstlisting}[language=Java]
int elapsedTimeInDays;
int daysSinceCreation;
int daysSinceModification;
int fileAgeInDays;
\end{lstlisting}


\end{frame}
%------------------------------------------------------------------------

%------------------------------------------------------------------------
\begin{frame}[fragile]{Avoid Disinformative Names}


Avoid names with baggage, unless you want the baggage.
\begin{itemize}
\item {\tt hp} not a good name for hypotenuse.  hp could also be Hewlett-Packard or horsepower.
\end{itemize}

Don't hint at implementation details in a variable name.
\begin{itemize}
\item Prefer {\tt accounts} to {\tt accountList}.
\item Note: certainly do want to indicate that a variable is a collection by giving it a plural name.
\end{itemize}

Superbad: using O, 0, l, and 1.
\begin{lstlisting}[language=Java]
int a = l;
if ( O == l )
  a=O1;
else
  l=01;
\end{lstlisting}
Don't think you'll never see code like this.  Sadly, you will.

\end{frame}
%------------------------------------------------------------------------


%------------------------------------------------------------------------
\begin{frame}[fragile]{Avoid Encodings}


Modern type systems and programming tools make encodings even more unnecessary.  So, AVOID ENCODINGS!  Consider:
\begin{lstlisting}[language=Java]
public class Part {
  private String m_dsc; // The textual descriptio
  void setName(String name) {
    m_dsc = name;
  }
}
\end{lstlisting}
The {\tt m\_} is useless clutter.  Much better to write:
\begin{lstlisting}[language=Java]
public class Part {
  private String description;
  void setDescription(String description) {
    this.description = description;
  }
}
\end{lstlisting}


\end{frame}
%------------------------------------------------------------------------

\section{Clean Fuctions}

%------------------------------------------------------------------------
\begin{frame}[fragile]{Functions Should be Small and Do one Thing Only}


\begin{itemize}
\item The first rule of functions: functions should be small.
\item The second rule of functions: functions should be small.
\end{itemize}
How small?  A few lines, 5 or 10.  ``A screen-full'' is no longer meaningful with large monitors and small fonts.

Some signs a function is doing too much:
\begin{itemize}
\item Many parameters.  ``If you have a procedure with ten parameters, you probably missed some.'' -- Perlis
\item ``Sections'' within a function, often delimited by blank lines.
\item Deeply nested logic.
\end{itemize}

\end{frame}
%------------------------------------------------------------------------

%------------------------------------------------------------------------
\begin{frame}[fragile]{Writing Functions that Do One Thing}


\begin{itemize}
\item One level of abstraction per function.
\begin{itemize}
\item A function that implements a higher-level algorithm should call helper functions to execute the steps of the algorithm.
\end{itemize}
\item Write code using the stepdown rule.
\begin{itemize}
\item Code should read like a narrative from top to bottom.
\item Read a higher level function to get the big picture, the functions below it to get the details.
\end{itemize}

\end{itemize}


\end{frame}
%------------------------------------------------------------------------

%------------------------------------------------------------------------
\begin{frame}[fragile]{Function Parameters}


Common monadic (one argument) forms
\begin{itemize}
\item Predicate functions: {\tt boolean fileExists(``MyFile'')}
\item Transformations: {\tt InputStream fileOpen(``MyFile'')}
\item Events: {\tt void passwordAttemptFailedNtimes(int attempts)}
\end{itemize}

Dyadic, triadic, and higher numbers of function arguments are much harder to get right.  Even one argument functions are problematic.  Consider flag argumets:
\begin{itemize}
\item Instead of {\tt render(boolean isSuite)}, a call to which would look like {\tt render(true)}, 
\item write two methods, like {\tt renderForSuite()} and {\tt renderForSingleTest()}
\end{itemize}

Keep in mind that in OOP, every instance method call has an implicit argument: the object on which it is invoked.
\end{frame}
%------------------------------------------------------------------------

%------------------------------------------------------------------------
\begin{frame}[fragile]{Minimizing the Number of Arguments}


Use objects.  Instead of
\begin{lstlisting}[language=Java]
public void doSomethingWithEmployee(String name, double pay, Date hireDate)
\end{lstlisting}
Represent emplyees with a class:
\begin{lstlisting}[language=Java]
public void doSomethingWith(Employee employee)
\end{lstlisting}

Use argument lists
\begin{lstlisting}[language=Java]
public int max(int ... numbers)
public String format(String format, Object... args)
\end{lstlisting}


\end{frame}
%------------------------------------------------------------------------
%------------------------------------------------------------------------
\begin{frame}[fragile]{Have no Side Effects}


The checkPassword method below:
\begin{lstlisting}[language=Java]
public class UserValidator {
  private Cryptographer cryptographer;
  public boolean checkPassword(String userName, String password) {
    User user =  UserGateway.findByName(userName);
    if (user != User.NULL) {
      String codedPhrase = user.getPhraseEncodedByPassword();
      String phrase = cryptographer.decrypt(codedPhrase, password);
      if ("Valid Password".equals(phrase)) {
        Session.initialize();
        return true; }
    }
    return false; }
}
\end{lstlisting}
Has the side effect of initializing the session.
\begin{itemize}
\item Might erase an existing session, or
\item might create temporal coupling: can only check password for user that doesn't have an existing session.
\end{itemize}


\end{frame}
%------------------------------------------------------------------------

\section{Clean Comments}

%------------------------------------------------------------------------
\begin{frame}[fragile]{Clean Comments}


Comments are (usually) evil.
\begin{itemize}
\item Most comments are compensation for failures to express ideas in code.
\item Comments become baggage when chunks of code move.
\item Comments become stale when code changes.
\item Result: comments lie.
\end{itemize}

Comments don't make up for bad code.  If you feel the need for a comment to explain some code, put effort into improving the code, not authoring comments for it.


\end{frame}
%------------------------------------------------------------------------

%------------------------------------------------------------------------
\begin{frame}[fragile]{Good Names Can Obviate Comments}


\begin{lstlisting}[language=Java]
// Check to see if the employee is eligible for full benefits
if ((employee.flags & HOURLY_FLAG) && (employee.age > 65))
\end{lstlisting}
We're representing a business rule as a boolean expression and naming it in a comment.  Use the language to express this idea:
\begin{lstlisting}[language=Java]
if (employee.isEligibleForFullBenefits())
\end{lstlisting}
Now if the business rule changes, we know exactly where to change the code that represents it, and the code can be reused.  (What does ``reused'' mean?)\\
%% Another example:
%% \begin{lstlisting}[language=Java]
%% // Returns an instance of the Responder being tested.
%% protected abstract Responder responderInstance();
%% \end{lstlisting}
%% Get rid of the comment and use the language:
%% \begin{lstlisting}[language=Java]
%% protected abstract Responder responderBeingTested;
%% \end{lstlisting}


\end{frame}
%------------------------------------------------------------------------

\section{Clean Formatting}

%------------------------------------------------------------------------
\begin{frame}[fragile]{Clean Formatting}

\begin{quote}
Code should be written for human beings to understand, and only incidentally for machines to execute. -- Hal Abelson and Gerald Sussman, SICP
\end{quote}

\begin{quote}
The purpose of a computer program is to tell other people what you want the computer to do. -- Donald Knuth
\end{quote}


The purpose of formatting is to facilitate communication.  The formatting of code conveys information to the reader.



\end{frame}
%------------------------------------------------------------------------

%------------------------------------------------------------------------
\begin{frame}[fragile]{Vertical Formatting}

\begin{itemize}
\item Newspaper metaphor
\item Vertical openness between concepts
\item Vertical density
\item Vertical distance
\item Vertical ordering
\end{itemize}


\end{frame}
%------------------------------------------------------------------------

%------------------------------------------------------------------------
\begin{frame}[fragile]{Vertical Openness Between Concepts}


Notice how vertical openness helps us locate concepts in the code more quickly.
\begin{lstlisting}[language=Java]
package fitnesse.wikitext.widgets;

import java.util.regex.*;

public class BoldWidget extends ParentWidget {
  public static final String REGEXP = "'''.+?'''";
  private static final Pattern pattern = Pattern.compile("'''(.+?)'''",
    Pattern.MULTILINE + Pattern.DOTALL
  );

  public BoldWidget(ParentWidget parent, String text) throws Exception { 
    super(parent);
    Matcher match = pattern.matcher(text);
    match.find();
    addChildWidgets(match.group(1));
  }
}
\end{lstlisting}

\end{frame}
%------------------------------------------------------------------------

%------------------------------------------------------------------------
\begin{frame}[fragile]{Vertical Openness Between Concepts}


If we leave out the blank lines:
\vspace{-.05in}
\begin{lstlisting}[language=Java]
package fitnesse.wikitext.widgets;
import java.util.regex.*;
public class BoldWidget extends ParentWidget {
  public static final String REGEXP = "'''.+?'''";
  private static final Pattern pattern = Pattern.compile("'''(.+?)'''",
    Pattern.MULTILINE + Pattern.DOTALL
  );
  public BoldWidget(ParentWidget parent, String text) throws Exception { 
    super(parent);
    Matcher match = pattern.matcher(text);
    match.find();
    addChildWidgets(match.group(1));
  }
}
\end{lstlisting}
\vspace{-.05in}
\begin{itemize}
\item It's harder to distinguish the package statement, the beginning and end of the imports, and the class declaration.
\item It's harder to locate where the instance variables end and methods begin.
\end{itemize}

\end{frame}
%------------------------------------------------------------------------

%------------------------------------------------------------------------
\begin{frame}[fragile]{Vertical Density}

\vspace{-.05in}
Openness separates concepts. Density implies association. Consider:
\vspace{-.25in}
\begin{lstlisting}[language=Java]
public class ReporterConfig {
  /** The class name of the reporter listener */
  private String className;

  /** The properties of the reporter listener */
  private List<Property> properties = new ArrayList<Property>();

  public void addProperty(Property property) {
    properties.add(property);
  }
\end{lstlisting}
\vspace{-.05in}
The vertical openness (and bad comments) misleads the reader.  Better to use closeness to convey relatedness:
\vspace{-.05in}
\begin{lstlisting}[language=Java]
public class ReporterConfig {
  private String className;
  private List<Property> properties = new ArrayList<Property>();

  public void addProperty(Property property) {
    properties.add(property);
  }
}
\end{lstlisting}

\end{frame}
%------------------------------------------------------------------------

%------------------------------------------------------------------------
\begin{frame}[fragile]{Vertical Distance and Ordering}


Concepts that are closely related should be vertically close to each other.
\begin{itemize}
\item Variables should be declared as close to their usage as possible.
\item Instance variables should be declared at the top of the class.
\item Dependent functions: callers should be above callees.
\end{itemize}


\end{frame}
%------------------------------------------------------------------------

%------------------------------------------------------------------------
\begin{frame}[fragile]{Horizontal Openness and Density}

\begin{itemize}
\item Keep lines short.  Uncle Bob says 120, but he's wrong.  Keep your lines at 80 characters or fewer if possible (sometimes it is impossible, but very rarely).
\item Put spaces around {\tt =} to accentuate the distinction between the LHS and RHS.
\item Don't put spaces between method names and parens, or parens and paramter lists - they're closely related, so should be close.
\item Use spaces to accentuate operator precedence, e.g., no space between unary operators and their operands, space between binary operators and their operands.
\item Don't try to horizontally align lists of assignments -- it draws attention to the wrong thing and can be misleading, e.g., encouraging the reader to read down a column.
\item Always indent scopes (classes, methods, blocks).
\end{itemize}


\end{frame}
%------------------------------------------------------------------------

%------------------------------------------------------------------------
\begin{frame}[fragile]{Team Rules}

\begin{itemize}
\item Every team should agree on a coding standard and everyone should adhere to it.
\item Don't modify a file just to change the formatting, but if you are modifying it anyway, go ahead and fix the formatting of the code you modify.
\item Code formatting standards tend to get religious.  My rule: make your code look like the language inventor's code.
\item If the language you're using has a code convention (like Java's), use it!
\end{itemize}


\end{frame}
%------------------------------------------------------------------------


% %------------------------------------------------------------------------
% \begin{frame}[fragile]{}


% \begin{lstlisting}[language=Java]

% \end{lstlisting}

% \begin{itemize}
% \item
% \end{itemize}


% \end{frame}
% %------------------------------------------------------------------------


\end{document}
