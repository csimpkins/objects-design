\documentclass{beamer}

\newcommand{\course}{CS 2340 Objects and Design}
\newcommand{\lesson}{Software Design}
\newcommand{\code}{http://www.cc.gatech.edu/~simpkins/teaching/gatech/cs2340/code}

\author[Chris Simpkins] 
{Christopher Simpkins \\\texttt{chris.simpkins@gatech.edu}}
\institute[Georgia Tech] % (optional, but mostly needed)

\date[CS 1331]{}

\include{beamer-common}

% \beamerdefaultoverlayspecification{<+->}


\begin{document}

\begin{frame}
  \titlepage
\end{frame}



%------------------------------------------------------------------------
\begin{frame}[fragile]{Design}


Design (noun)
\begin{quote}
A plan or protocol for carrying out or accomplishing something. -- Webster's Dictionary
\end{quote}

Engineering design (verb):
\begin{quote}
A systematic, intelligent process in which designers generate, evaluate and specify designs for devices, systems, or processes whose form(s) and function(s) acheive clients' objectives and users' needs while satisfying a specified set of constraints. -- Dym and Little, quoted in Carlos Otero, Software Engineering Design
\end{quote}


\end{frame}
%------------------------------------------------------------------------

%------------------------------------------------------------------------
\begin{frame}[fragile]{Fundamental Design Principles -- Otero}

\begin{itemize}
\item Modularization
\item Abstraction
\item Encapsulation
\item Coupling and Cohesion
\item Separation of interface and implementation
\item Suficiency and completeness
\end{itemize}


\end{frame}
%------------------------------------------------------------------------

%------------------------------------------------------------------------
\begin{frame}[fragile]{Object Design -- Wirfs-Brock}


\begin{itemize}
\item Design driven by roles and responsibilities
\item Write story about system, identify themes and abstractions, identify candidate roles/classes that support themes
\item When candidates identified, model reponsibilities and collaborations
\item Exploratory design with CRC cards (candidates, responsibilities, collaborations)
\item Object roles often fit into these steriotypes:
  \begin{itemize}
  \item Informatoin holder -- knows and provides information
  \item Structurer -- maintains relationships between objects and information about those relationships
  \item Service provider -- performs work and, in general, offers computing services
  \item Coordinator -- reacts to events by delegating tasks to others
  \item Controller -- makes decisions and closely directs others' actions
  \item Interfacer -- transofrms information and requests between idstinct parts of the system
  \end{itemize}
\end{itemize}


\end{frame}
%------------------------------------------------------------------------

%------------------------------------------------------------------------
\begin{frame}[fragile]{Agile Design -- Martin}


Fundamental principle of agile design:
\begin{quote}
The code is the design. -- Jack Reeves, 1992
\end{quote}

Design Smells
\begin{itemize}
\item Rigidity -- system is too hard to change becuase change in one place forces changes in many other places
\item Fragility -- changes break things that are conceptually unrelated
\item Immobility -- too hard to resuse components in other systems
\item Viscosity -- hard to do it right, easy to do it wrong
\item Needless Complexity -- infrastructure with no direct benefit
\item Needless Repetition -- repeated structures that could be unified under a single abstraction
\item Opacity -- hard to read and understand
\end{itemize}
Design smells avoided or fixed by applying design principles like SRP, OCP ...

\end{frame}
%------------------------------------------------------------------------

%------------------------------------------------------------------------
\begin{frame}[fragile]{Levels of Design}

\begin{itemize}
\item High-level: system architecture
\item Detailed Design
\end{itemize}


\end{frame}
%------------------------------------------------------------------------

%------------------------------------------------------------------------
\begin{frame}[fragile]{System Architectures}

\begin{itemize}
\item Stand-alone
\item Client-server
\item N-tier
\item Service-oriented architecture
\end{itemize}

\end{frame}
%------------------------------------------------------------------------

%% %------------------------------------------------------------------------
%% \begin{frame}[fragile]{Client-server Architecture}


%% \begin{lstlisting}[language=Java]

%% \end{lstlisting}

%% \begin{itemize}
%% \item
%% \end{itemize}


%% \end{frame}
%% %------------------------------------------------------------------------

%% %------------------------------------------------------------------------
%% \begin{frame}[fragile]{Service-oriented Architecture}


%% \begin{lstlisting}[language=Java]

%% \end{lstlisting}

%% \begin{itemize}
%% \item
%% \end{itemize}


%% \end{frame}
%% %------------------------------------------------------------------------

%------------------------------------------------------------------------
\begin{frame}[fragile]{Detailed Design}


\begin{itemize}
\item Classes and methods
\item Data interchange formats (XML schema, JSON)
\item Entity-relationship models, database schema
\end{itemize}


\end{frame}
%------------------------------------------------------------------------

%------------------------------------------------------------------------
\begin{frame}[fragile]{Unified Modeling Langauge (UML)}


A standardized diagrammatic language for communicating OO designs in a language-independent way.  Very rich, but for now focus on:
\begin{itemize}
\item use cases,
\item domain model (classes and associations),
\item packages, and
\item sequence digrams.
\end{itemize}


\end{frame}
%------------------------------------------------------------------------

%------------------------------------------------------------------------
\begin{frame}[fragile]{Use Cases}


\begin{center}
\includegraphics[width=3in]{use-case-create-todo.png}
\end{center}

A use case describes some user's interaction with the system.  Most use cases contain:
\begin{itemize}
\item an {\it actor}, here simply ``User,''
\item a {\it use case}, here ``Create new todo,'' and
\item (optionally) a {\it system boundary}, here ``tomcat-todo.''
\end{itemize}


\end{frame}
%------------------------------------------------------------------------

%------------------------------------------------------------------------
\begin{frame}[fragile]{Class Diagrams}


\begin{center}
\includegraphics[width=2in]{uml-class-todo.png}
\end{center}
\vspace{-.25in}
Class diagrams contain
\begin{itemize}
\item a {\it class name}, here ``Todo,''
\item {\it instance variables}, here ``title'' and ``task''.  Note that types are given after names, as in ``: String''.  The ``-'' means private.
\item {\it methods}.  The ``+'' means public. (``\#'' means protected, but we have no protected members in this example.)
\end{itemize}


\end{frame}
%------------------------------------------------------------------------

%------------------------------------------------------------------------
\begin{frame}[fragile]{Associations}


\begin{center}
\includegraphics[width=3in]{uml-association-todo.png}
\end{center}

\begin{itemize}
\item {\it TodoDb} is italicized, meaning it is abstract.  The ``<<interface>>'' further means that it is an interface.
\item {\tt TodoDbHashMapImpl} is a subtype of {\tt TodoDb}.
\item {\tt TodoDbHashMapImpl} is {\it composed} of a {\tt HashMap<K,V>}.
\item {\tt HashMap} is a paramterized type with type parameters K and V.
\item {\tt TodoDb} aggregates {\tt Todo} objects.
\end{itemize}


\end{frame}
%------------------------------------------------------------------------

%------------------------------------------------------------------------
\begin{frame}[fragile]{Packages}


\begin{center}
\includegraphics[width=3in]{uml-package-todo.png}
\end{center}

\begin{itemize}
\item The tab shows the package name.
\item The main box lists classes and interfaces using the same +, -, and \# visibility modifiers used for members in class diagrams.
\item An alternative form is to simply put the package name in the main box and not list the members of the package.
\end{itemize}


\end{frame}
%------------------------------------------------------------------------

%------------------------------------------------------------------------
\begin{frame}[fragile]{Sequence Diagrams}


\begin{center}
\includegraphics[width=4in]{uml-sequence-create-todo.png}
\end{center}

\begin{itemize}
\item The top rectangles represent objects (instances of types/classes).
\item Time progresses vertically downward.
\item The dashed lines represent object lifetimes.
\item The narrow vertical boxes represent operations of the objects (here, only create on the todoDb object).
\item Arrows are ``messages'' or method calls and returns.
\end{itemize}


\end{frame}
%------------------------------------------------------------------------

%------------------------------------------------------------------------
\begin{frame}[fragile]{Closing Thoughts}


\begin{itemize}
\item Design is an art born of empirical science and intuitive experience
\item Practical experience and formal studies of software systems have produced design principles and guidelines
\item A design is never ``perfect''
\item Design involves tradeoffs -- balancing constraints means prioritizing
\end{itemize}
We don't expect your projects to be perfectly designed, or even well-designed.  But we want you to apply design principles and processes and learn from your experience.

\end{frame}
%------------------------------------------------------------------------

% %------------------------------------------------------------------------
% \begin{frame}[fragile]{}


% \begin{lstlisting}[language=Java]

% \end{lstlisting}

% \begin{itemize}
% \item
% \end{itemize}


% \end{frame}
% %------------------------------------------------------------------------


\end{document}
