\documentclass{beamer}

\newcommand{\course}{CS 2340 Objects and Design}
\newcommand{\lesson}{Pro Java}
\newcommand{\code}{http://www.cc.gatech.edu/~simpkins/teaching/gatech/cs2340/code}

\author[Chris Simpkins]
{Christopher Simpkins \\\texttt{chris.simpkins@gatech.edu}}
\institute[Georgia Tech] % (optional, but mostly needed)

\date[CS 1331]{}

\subject{\lesson}


% If you have a file called "university-logo-filename.xxx", where xxx
% is a graphic format that can be processed by latex or pdflatex,
% resp., then you can add a logo as follows:

% \pgfdeclareimage[width=0.6in]{coc-logo}{cc_2012_logo}
% \logo{\pgfuseimage{coc-logo}}

\mode<presentation>
{
  \usetheme{Berlin}
  \useoutertheme{infolines}

  % or ...

 \setbeamercovered{transparent}
  % or whatever (possibly just delete it)
}

\usepackage{hyperref}
\usepackage{fancybox}
\usepackage{listings}
\usepackage[abbr]{harvard}

\usepackage[english]{babel}
% or whatever

\usepackage[latin1]{inputenc}
% or whatever

\usepackage{times}
\usepackage[T1]{fontenc}
% Or whatever. Note that the encoding and the font should match. If T1
% does not look nice, try deleting the line with the fontenc.


\usepackage{listings}
 
% "define" Scala
\lstdefinelanguage{scala}{
  morekeywords={abstract,case,catch,class,def,%
    do,else,extends,false,final,finally,%
    for,if,implicit,import,match,mixin,%
    new,null,object,override,package,%
    private,protected,requires,return,sealed,%
    super,this,throw,trait,true,try,%
    type,val,var,while,with,yield},
  otherkeywords={=>,<-,<\%,<:,>:,\#,@},
  sensitive=true,
  morecomment=[l]{//},
  morecomment=[n]{/*}{*/},
  morestring=[b]",
  morestring=[b]',
  morestring=[b]""",
}

\usepackage{color}
\definecolor{dkgreen}{rgb}{0,0.6,0}
\definecolor{gray}{rgb}{0.5,0.5,0.5}
\definecolor{mauve}{rgb}{0.58,0,0.82}
 
% Default settings for code listings
\lstset{frame=tb,
  language=scala,
  aboveskip=2mm,
  belowskip=2mm,
  showstringspaces=false,
  columns=flexible,
  basicstyle={\scriptsize\ttfamily},
  numbers=none,
  numberstyle=\tiny\color{gray},
  keywordstyle=\color{blue},
  commentstyle=\color{dkgreen},
  stringstyle=\color{mauve},
  frame=single,
  breaklines=true,
  breakatwhitespace=true,
  keepspaces=true
  %tabsize=3
}


\title[\course] % (optional, use only with long
                                      % paper titles)
{\lesson}

\subtitle{}
%% {Include Only If Paper Has a Subtitle}

\newcommand{\link}[2]{\href{#1}{\textcolor{blue}{\underline{#2}}}}


% \beamerdefaultoverlayspecification{<+->}


\begin{document}

\begin{frame}
  \titlepage
\end{frame}

%------------------------------------------------------------------------
\begin{frame}[fragile]{Pro Java}


You know the basics of Java.  Today you'll learn a few baics properties of professional Java projects, including
\begin{itemize}
\item the classpath,
\item separating source and compiler output,
\item project directory layout,
\item packages,
\item jar files, and
\item using an IDE.
\end{itemize}

\end{frame}
%------------------------------------------------------------------------

%------------------------------------------------------------------------
\begin{frame}[fragile]{The Classpath}


Just as your operating system shell looks in the {\tt PATH} environment variable for executable files, JDK tools (such as {\tt javac} and {\tt java}) look in the {\tt CLASSPATH} for Java classes. To specify a classpath:

\begin{itemize}
\item set an environment variable named {\tt CLASSAPTH}, or
\item specify a classpath on a per-application basis by using the {\tt -cp} switch.  The classpath set with {\tt -cp} overrides the {\tt CLASSPATH} environment variable.
\end{itemize}
Don't use the {\tt CLASSPATH} environment variable.  If it's already set, clear it with (on Windows):
\begin{lstlisting}[language=Java]
C:> set CLASSPATH=
\end{lstlisting}
or (on Unix):
\begin{lstlisting}[language=Java]
$ unset CLASSPATH
\end{lstlisting}

\end{frame}
%------------------------------------------------------------------------

%------------------------------------------------------------------------
\begin{frame}[fragile]{Specifying a Classpath}

\vspace{-.05in}
A classpath specification is a list of places to find {\tt .class} files and other resources.  Two kinds of elements in this list:

\begin{itemize}
\item directories in which to find {\tt .class} files on the filesystem, or
\item {\tt .jar} files that contain archives of directory trees containing {\tt .class} files and other files (more later).
\end{itemize}

To compile and run a program with compiler output ({\tt .class} files) in the current directory and a library Jar file in the {\tt lib} directory called {\tt util.jar}, you'd specify the classpath like this:
\vspace{-.05in}
\begin{lstlisting}[language=bash]
$ ls -R # -R means recursive (show subdirectory listings)
MyProgram.java     AnotherClass.java

./lib:
util.jar
$ javac -cp .:lib/util.jar *.java # : separates classpath elements
$ java -cp .:lib/util.jar MyProgram # would be ; on Windows
\end{lstlisting}
\vspace{-.05in}
Notice that you include the entire classpath in the {\tt -cp}, which includes the current directory ({\tt .} means ``current directory'').

\end{frame}
%------------------------------------------------------------------------

%------------------------------------------------------------------------
\begin{frame}[fragile]{Separating Source and Compiler Output}


To reduce clutter, you can compile classes to another directory with {\tt -d} option to {\tt javac}
\begin{lstlisting}[language=Java]
$ mkdir classes
$ javac -d classes HelloWorld.java
$ ls classes/
HelloWorld.class
\end{lstlisting}
Specify classpath for an application with the {\tt -cp} option to {\tt
  java}.
\begin{lstlisting}[language=Java]
$ java -cp ./classes HelloWorld
Hello, world!
\end{lstlisting}

If you really want to keep your project's root directory clean (and you do), you can put your source code in another directory too, like {\tt src}.
\begin{lstlisting}[language=Java]
$ mkdir src
$ mv HelloWorld.java src/
$ javac -d ./classes src/HelloWorld.java
$ java -cp ./classes HelloWorld
Hello, world!
\end{lstlisting}

\end{frame}
%------------------------------------------------------------------------

%------------------------------------------------------------------------
\begin{frame}[fragile]{Project Directory Layout}


Source Directories
\begin{itemize}
\item {\tt src/main/java} for Java source files
\item {\tt src/main/resources} for resources that will go on the classpath, like image files
\end{itemize}

Output Directories
\begin{itemize}
\item {\tt target/classes} for compiled Java .class files and resources copied from {\tt src/main/resources}
\end{itemize}

There's more, but this is enough for now.  More details on the de-facto standard Java project directory layout can be found at \link{http://maven.apache.org/guides/introduction/introduction-to-the-standard-directory-layout.html}{http://maven.apache.org/guides/introduction/introduction-to-the-standard-directory-layout.html}

\end{frame}
%------------------------------------------------------------------------

%------------------------------------------------------------------------
\begin{frame}[fragile]{Organizing your Code in Packages}


All professional Java projects organize their code in packages.  The standard package naming scheme is to use reverse domain name, followed by project specific packages.  So if you're writing a zombie game and you're in the Lab for Interactive AI your application's base package would be specified like this (first line of source files):
\vspace{-.05in}
\begin{lstlisting}[language=Java]
package edu.gatech.iai.zombie;
\end{lstlisting}
and it would be located in a directory under your {\tt src/main/java} directory as follows
\begin{lstlisting}[language=Java]
src/main/java/edu/gatech/iai/zombie
\end{lstlisting}
\vspace{-.05in}
And if you tell {\tt javac} to put compiler output in {\tt target/classes} then the compiled {\tt .class} file would end up in:
\begin{lstlisting}[language=Java]
target/classes/edu/gatech/iai/zombie
\end{lstlisting}

\end{frame}
%------------------------------------------------------------------------

%------------------------------------------------------------------------
\begin{frame}[fragile]{Jar Files}

A jar archive, or jar file, is a Zip-formatted archive of a directory tree.
Java uses jar files as a distribution format for libraries.

\begin{itemize}
\item To create a JAR file: {\tt jar cf jar-file input-file(s)}
\item To view the contents of a JAR file {\tt jar tf jar-file}
\item To extract the contents of a JAR file: {\tt jar xf jar-file} or {\tt unzip jar-file}
\item To extract specific files from a JAR file: {\tt jar xf jar-file archived-file(s)}
\item To run an application packaged as a JAR file (requires the Main-class manifest header): {\tt java -jar app.jar}
\end{itemize}

See
\link{http://docs.oracle.com/javase/tutorial/deployment/jar/index.html}{http://docs.oracle.com/javase/tutorial/deployment/jar/index.html} for more details.

\end{frame}
%------------------------------------------------------------------------

%------------------------------------------------------------------------
\begin{frame}[fragile]{Reorganizing Blackjack}

Let's apply these organizational practices to an existing application.
\begin{itemize}
\item Create a directory somewhere on your hard disk called, say, {\tt blackjack}
\item Download a \link{www.cc.gatech.edu/~simpkins/teaching/gatech/cs2340/code/blackjack/blackjack.zip}{zip file of the Blackjack source} to your newly created project directory.  If you have {\tt wget} installed on your computer\footnote{On a mac with \link{http://brew.sh/}{Homebrew} you can just do {\tt brew install wget}.} you can go to your project directory on the command line and do:
\begin{lstlisting}[language=Java]
$ wget www.cc.gatech.edu/~simpkins/teaching/gatech/cs2340/code/blackjack/blackjack.zip
\end{lstlisting}
\end{itemize}

\end{frame}
%------------------------------------------------------------------------

%------------------------------------------------------------------------
\begin{frame}[fragile]{Dealing with Zip/Jar Files}

Before we unzip a zip file it's a good idea to see what's in it, which is easy with {\tt jar}:
\begin{lstlisting}[language=Java]
$ jar tf blackjack.zip
Blackjack.java
BlackjackHand.java
BlackjackPlayer.java
Deck.java
HumanPlayer.java
PlayingCard.java
RandomPlayer.java
\end{lstlisting}
Notice that unarchiving this zip file will not create a subdirectory, so we need to do that ourselves (which we already have -- {\tt blackjack/})

\begin{lstlisting}[language=Java]
$ mkdir blackjack
$ mv blackjack.zip blackjack/
$ cd blackjack/
$ unzip blackjack.zip
Archive:  blackjack.zip
  inflating: Blackjack.java
  inflating: BlackjackHand.java
  inflating: BlackjackPlayer.java
  inflating: Deck.java
  inflating: HumanPlayer.java
  inflating: PlayingCard.java
  inflating: RandomPlayer.java
\end{lstlisting}

Now we're ready to reorganize the project source.

\end{frame}
%------------------------------------------------------------------------

%------------------------------------------------------------------------
\begin{frame}[fragile]{Package Statements}

Before we move these soruce files into the project directory structure (which is somewhat annoyingly deeply nested) we can add package statements.  First, decide on a name:

\begin{itemize}
\item We're part of Georgia Tech, so our package name should begin with {\tt edu.gatech}
\item Our "sub-organization" within GT is CS 2340, so we'll add {\tt cs2340} to the package name
\item Our application name is Blackjack, so that can be the final piece of our package name.
\end{itemize}

All of this yields a package name of\\
\vspace{.1in}
\begin{center}
{\tt edu.gatech.cs2340.blackjack}
\end{center}

\end{frame}
%------------------------------------------------------------------------

%------------------------------------------------------------------------
\begin{frame}[fragile]{Updating Source Files}

We need to add a package statement to the top of each of our source files (we'll put everythign in one package).

We can do this with a Unix one-liner:
\begin{lstlisting}[langauge=bash]
$ for file in `ls *.java`; \
    do printf "package edu.gatech.cs2340.blackjack;\n\n" > $file.new; \
    cat $file >> $file.new; \
    mv $file.new `basename $file .new`; \
  done
\end{lstlisting}
OK, it's a bit of a stretch to call that a one-liner, and I'm sure a real Unix geek could do it more elegantly, but it works.

\begin{lstlisting}[language=Java]
$ head -n 5 Blackjack.java
package edu.gatech.cs2340.blackjack;

import java.util.Scanner;

public class Blackjack {
\end{lstlisting}

\end{frame}
%------------------------------------------------------------------------

%------------------------------------------------------------------------
\begin{frame}[fragile]{Creating the Directory Structure}

Remember:
\begin{itemize}
\item The root of the source directory is {\tt src/main/java}
\item The source files go in a directory structure that matches the package name {\it under} the source root
\end{itemize}

Use {\tt mkdir -p} to create the whole nested directory structure:
\begin{lstlisting}[language=bash]
$ mkdir -p src/main/java/edu/gatech/cs2340/blackjack
$ mv *.java src/main/java/edu/gatech/cs2340/blackjack/
[chris@nijinsky ~/scratch/blackjack]
$ ls -R
blackjack.zip src

[empty intermediate directories elided]

./src/main/java/edu/gatech/cs2340/blackjack:
Blackjack.java    Deck.java   RandomPlayer.java
BlackjackHand.java  HumanPlayer.java
BlackjackPlayer.java  PlayingCard.java


\end{lstlisting}

\end{frame}
%------------------------------------------------------------------------

%------------------------------------------------------------------------
\begin{frame}[fragile]{Compiling and Running}

We want compiler output to go into {\tt target/classes}, so we must create that directory:
\begin{lstlisting}[language=Java]
$ mkdir -p target/classes
\end{lstlisting}

Then we can compile and run from the command line:

\begin{lstlisting}[language=Java]
$ javac -d target/classes/ -cp target/classes/ src/main/java/edu/gatech/cs2340/blackjack/*.java
[chris@nijinsky ~/scratch/blackjack]
$ java -cp target/classes/ edu.gatech.cs2340.blackjack.Blackjack
What's your name? Chris
...
\end{lstlisting}

It's useful to know how to do this so you can debug problem with your IDE or build script, but you'll normally set up an IDE and an automated biuld.

\end{frame}
%------------------------------------------------------------------------

%------------------------------------------------------------------------
\begin{frame}[fragile]{Using an IDE}


Here are a few basic things you need to configrue when using an IDE:
\begin{itemize}
\item Editor settings for non-awful source code
\item Source Directory
\item Classpath
\item Libraries
\end{itemize}
The best approach to most of this is to generate an IDE project configuration from your build specification, e.g., {\tt build.xml}.  Let's see how to do these things with Eclipse.

\end{frame}
%------------------------------------------------------------------------


%------------------------------------------------------------------------
\begin{frame}[fragile]{Closing Thoughts}


\begin{itemize}
\item  There's more ``Pro Java'' to learn, like Junit and Checkstyle, but these are the basics.
\item Speaking of Checkstyle, follow the Java code conventions at \link{http://www.oracle.com/technetwork/java/codeconv-138413.html}{http://www.oracle.com/technetwork/java/codeconv-138413.html}.
\item We'll learn much more about build automation and Ant.
\end{itemize}


\end{frame}
%------------------------------------------------------------------------

% %------------------------------------------------------------------------
% \begin{frame}[fragile]{}

% \begin{itemize}
% \item
% \end{itemize}

% \begin{lstlisting}[language=Java]

% \end{lstlisting}

% \end{frame}
% %------------------------------------------------------------------------



\end{document}
