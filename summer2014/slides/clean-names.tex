\documentclass{beamer}

\newcommand{\course}{CS 2340 Objects and Design}
\newcommand{\lesson}{Clean Names}
\newcommand{\code}{http://www.cc.gatech.edu/~simpkins/teaching/gatech/cs2340/code}

\author[Chris Simpkins] 
{Christopher Simpkins \\\texttt{chris.simpkins@gatech.edu}}
\institute[Georgia Tech] % (optional, but mostly needed)

\date[CS 1331]{}

\subject{\lesson}


% If you have a file called "university-logo-filename.xxx", where xxx
% is a graphic format that can be processed by latex or pdflatex,
% resp., then you can add a logo as follows:

% \pgfdeclareimage[width=0.6in]{coc-logo}{cc_2012_logo}
% \logo{\pgfuseimage{coc-logo}}

\mode<presentation>
{
  \usetheme{Berlin}
  \useoutertheme{infolines}

  % or ...

 \setbeamercovered{transparent}
  % or whatever (possibly just delete it)
}

\usepackage{hyperref}
\usepackage{fancybox}
\usepackage{listings}
\usepackage[abbr]{harvard}

\usepackage[english]{babel}
% or whatever

\usepackage[latin1]{inputenc}
% or whatever

\usepackage{times}
\usepackage[T1]{fontenc}
% Or whatever. Note that the encoding and the font should match. If T1
% does not look nice, try deleting the line with the fontenc.


\usepackage{listings}
 
% "define" Scala
\lstdefinelanguage{scala}{
  morekeywords={abstract,case,catch,class,def,%
    do,else,extends,false,final,finally,%
    for,if,implicit,import,match,mixin,%
    new,null,object,override,package,%
    private,protected,requires,return,sealed,%
    super,this,throw,trait,true,try,%
    type,val,var,while,with,yield},
  otherkeywords={=>,<-,<\%,<:,>:,\#,@},
  sensitive=true,
  morecomment=[l]{//},
  morecomment=[n]{/*}{*/},
  morestring=[b]",
  morestring=[b]',
  morestring=[b]""",
}

\usepackage{color}
\definecolor{dkgreen}{rgb}{0,0.6,0}
\definecolor{gray}{rgb}{0.5,0.5,0.5}
\definecolor{mauve}{rgb}{0.58,0,0.82}
 
% Default settings for code listings
\lstset{frame=tb,
  language=scala,
  aboveskip=2mm,
  belowskip=2mm,
  showstringspaces=false,
  columns=flexible,
  basicstyle={\scriptsize\ttfamily},
  numbers=none,
  numberstyle=\tiny\color{gray},
  keywordstyle=\color{blue},
  commentstyle=\color{dkgreen},
  stringstyle=\color{mauve},
  frame=single,
  breaklines=true,
  breakatwhitespace=true,
  keepspaces=true
  %tabsize=3
}


\title[\course] % (optional, use only with long
                                      % paper titles)
{\lesson}

\subtitle{}
%% {Include Only If Paper Has a Subtitle}

\newcommand{\link}[2]{\href{#1}{\textcolor{blue}{\underline{#2}}}}


% \beamerdefaultoverlayspecification{<+->}


\begin{document}

\begin{frame}
  \titlepage
\end{frame}



%------------------------------------------------------------------------
\begin{frame}[fragile]{Clean Code}


What is ``clean code?''
\begin{itemize}
\item Elegant and efficient. -- Bjarne Stroustrup
\item Simple and direct. Readable.  -- Grady Booch
\item Understandable by others, tested, literate. -- Dave Thomas
\item Code works pretty much as expected.  Beatuful code looks like the language was made for the problem. -- Ward Cunningham
\end{itemize}

Why do we care abou clean code?
\begin{itemize}
\item Messes are costly.  Quick and dirty to get it done ends up not getting it done and you will not enjoy it.  It's lose-lose!
\item We are professionals who care about our craft. 
\end{itemize}

The Boy Scout Rule

\end{frame}
%------------------------------------------------------------------------

%------------------------------------------------------------------------
\begin{frame}[fragile]{Meaningful Names}


\begin{itemize}
\item The name of a variable, method, or class should reveal its purpose.
\item If you feel the need to comment on the name itself, pick a better name.
\item Code with a dictionary close at hand.
\end{itemize}

Don't ever do this!
\begin{lstlisting}[language=Java]
int d; // elapsed time in days
\end{lstlisting}

Much better:
\begin{lstlisting}[language=Java]
int elapsedTimeInDays;
int daysSinceCreation;
int daysSinceModification;
int fileAgeInDays;
\end{lstlisting}


\end{frame}
%------------------------------------------------------------------------

%------------------------------------------------------------------------
\begin{frame}[fragile]{Intention-Revealing Names}


What is the purpose of this code?
\begin{lstlisting}[language=Java]
public List<int[]> getThem() {
  List<int[]> list1 = new ArrayList<int[]>();
  for (int[] x : theList)
    if (x[0] == 4)
      list1.add(x);
  return list1;
}
\end{lstlisting}
Why is it hard to tell? -- Code itself doesn't reveal context.

\begin{itemize}
\item What's in {\tt theList}?
\item What's special about item 0 in one of the arrays in {\tt theList}?
\item What does the magic number 4 represent?
\item What is client code supposed to do with the returned list?
\end{itemize}

\end{frame}
%------------------------------------------------------------------------

%------------------------------------------------------------------------
\begin{frame}[fragile]{Intention-Revealing Names Exercise}


Turns out, this code represents a game board for a mine sweeper game and {\tt theList} holds the cells of the game board.  Each cell is represented by and {\tt int[]} whose 0th element contains a status flag that means ``flagged.''\\
\vspace{.05in}
Look how much of a difference renaming makes:
\begin{lstlisting}[language=Java]
public List<int[]> getFlaggedCells() {
  List<int[]> flaggedCells = new ArrayList<int[]>();
  for (int[] cell : gameBoard)
    if (cell[STATUS_VALUE] == FLAGGED) flaggedCells.add(cell);
  return flaggedCells;
}
\end{lstlisting}
Even better, create a class to represent cells
\begin{lstlisting}[language=Java]
public List<Cell> getFlaggedCells() {
  List<Cell> flaggedCells = new ArrayList<Cell>();
  for (Cell cell : gameBoard)
    if (cell.isFlagged()) flaggedCells.add(cell);
  return flaggedCells;
}
\end{lstlisting}

\end{frame}
%------------------------------------------------------------------------

%------------------------------------------------------------------------
\begin{frame}[fragile]{Disinformative Names}


Avoid names with baggage, unless you want the baggage.
\begin{itemize}
\item {\tt hp} not a good name for hypotenuse.  hp could also be Hewlett-Packard or horsepower.
\end{itemize}

Don't hint at implementation details in a variable name.
\begin{itemize}
\item Prefer {\tt accounts} to {\tt accountList}.
\item Note: certainly do want to indicate that a variable is a collection by giving it a plural name.
\end{itemize}

Superbad: using O, 0, l, and 1.
\begin{lstlisting}[language=Java]
int a = l;
if ( O == l )
  a=O1;
else
  l=01;
\end{lstlisting}
Don't think you'll never see code like this.  Sadly, you will.

\end{frame}
%------------------------------------------------------------------------

%------------------------------------------------------------------------
\begin{frame}[fragile]{Names that Make Distinctions}


Consider this method header:
\begin{lstlisting}[language=Java]
public static void copyChars(char a1[], char a2[])
\end{lstlisting}
Which array is source?  WHich is desitination?  Make intention explicit:
\begin{lstlisting}[language=Java]
public static void copyChars(char source[], char desitination[])
\end{lstlisting}

Meaningless distinctions:
\begin{itemize}
\item {\tt ProductInfo} versus {\tt ProductData}
\item {\tt Customer} versus {\tt CustomerObject}
\end{itemize}
Don't be lazy with variable names.

\end{frame}
%------------------------------------------------------------------------

%------------------------------------------------------------------------
\begin{frame}[fragile]{Pronouncable and Searchable Names}


\begin{itemize}
\item You'll need to talk to other programmers about code, so use pronouncable names.
\item Also, using English words makes variable names easier to remember.
\item Using descriptive names also helps you search using tools like GREP.
\item Sometimes short names are acceptable if they are traditional.  For example {\tt i}, {\tt j} and {\tt k} for short nested loops.
\end{itemize}

General rule: the length of a variable name shoudl be proportional to its scope.

\end{frame}
%------------------------------------------------------------------------

%------------------------------------------------------------------------
\begin{frame}[fragile]{Encodings}


Some misguided programmers like to embed comments and type informatoin in variable names.
\begin{itemize}
\item In the bad old days of Windows programming in C Charles Simponyi, a hungarian programmer at Microsoft, created an encoding scheme for variable and function names.  For example, every long pointer to a null-terminated string was prefixed with {\tt lpsz} (long pointer string zero).
\item When Microsoft moved to ``C++'' for their MFC framework, they added encodings for member variables: the {\tt m\_} prefix (for ``member'').
\end{itemize}
Be very happy you never had to work with the Win API or MFC.  They were awful.

\end{frame}
%------------------------------------------------------------------------

%------------------------------------------------------------------------
\begin{frame}[fragile]{Avoid Encodings}


Modern type systems and programming tools make encodings even more unnecessary.  So, AVOID ENCODINGS!  Consider:
\begin{lstlisting}[language=Java]
public class Part {
  private String m_dsc; // The textual descriptio
  void setName(String name) {
    m_dsc = name;
  }
}
\end{lstlisting}
The {\tt m\_} is useless clutter.  Much better to write:
\begin{lstlisting}[language=Java]
public class Part {
  private String description;
  void setDescription(String description) {
    this.description = description;
  }
}
\end{lstlisting}


\end{frame}
%------------------------------------------------------------------------

%------------------------------------------------------------------------
\begin{frame}[fragile]{A Few Final Naming Guidelines}


\begin{itemize}
\item Avoid mental mapping. We're all smart.  Smart coders make things clear.
\begin{itemize}
\item So simple only a genius could have thought of it. -- Einstein
\item Simplicity does not precede complexity but follows it. -- Perlis
\end{itemize}
\item Use nouns or noun phrases for class names.
\item Use verbs or verb phrases for method names.
\item Don't use puns or jokes in names.
\item Use one word per concept.
\item Use CS terms in names.
\item Use problem domain terms in names.
\end{itemize}


\end{frame}
%------------------------------------------------------------------------

% %------------------------------------------------------------------------
% \begin{frame}[fragile]{}


% \begin{lstlisting}[language=Java]

% \end{lstlisting}

% \begin{itemize}
% \item
% \end{itemize}


% \end{frame}
% %------------------------------------------------------------------------


\end{document}
