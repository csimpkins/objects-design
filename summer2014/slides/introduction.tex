\documentclass{beamer}

\newcommand{\course}{CS 2340 Objects and Design}
\newcommand{\lesson}{Introduction}
\newcommand{\code}{http://www.cc.gatech.edu/~simpkins/teaching/gatech/cs2340/code}

\author[Chris Simpkins]
{Christopher Simpkins \\\texttt{chris.simpkins@gatech.edu}}
\institute[Georgia Tech] % (optional, but mostly needed)

\date[CS 2340]{}

\include{beamer-common}

% \beamerdefaultoverlayspecification{<+->}

\begin{document}

\begin{frame}
  \titlepage
\end{frame}

%------------------------------------------------------------------------
\begin{frame}[fragile]{Course Overview}

\begin{itemize}
\item Workload
\item Course Content
\item Syllabus
\end{itemize}

\end{frame}
%------------------------------------------------------------------------


%------------------------------------------------------------------------
\begin{frame}[fragile]{Expected Time Allotment}

\begin{quote}
One semester credit is expected to require at least three hours of scholarly activity per week.
\end{quote}
 -- \url{http://www.registrar.gatech.edu/faculty/fs_sch.php}\\
\vspace{.1in}
3 credit class = 9 hours a week (~12 in summer)\\
\vspace{.1in}
2.5 hours of lecture (3 x 50min, or 2 x 1:45 = 3.5 hours in summer)\\
\vspace{.1in}
At least 6.5 more hours (8.5 in summer) for reading, studying, and homeworks.

\end{frame}
%------------------------------------------------------------------------

%------------------------------------------------------------------------
\begin{frame}[fragile]{Your Semester Schedule}


\begin{quote}
One semester credit is expected to require at least three hours of scholarly activity per week.
\end{quote}
 -- \url{http://www.registrar.gatech.edu/faculty/fs_sch.php}\\
\vspace{.1in}
12 credit hours = 36 hours a week (49 hours in summer)\\
\vspace{.1in}
Full Time $\ge$ 12 credit hours (including summer)\footnote{\url{http://www.registrar.gatech.edu/students/semestersystem.php}}


\end{frame}
%------------------------------------------------------------------------

\section{Course Content}

%------------------------------------------------------------------------
\begin{frame}[fragile]{CS 2340}

Amalgam of two courses:
\begin{itemize}
\item Software engineering practicum
\begin{itemize}
\item Introduction to software engineering
\item Practical software engineering skills (tools, technologies, practices)
\item Preparation for design capstone (and real jobs)
\end{itemize}
\item Objects and Design
\begin{itemize}
\item Software design principles
\item Object-oriented design
\item Design patterns
\end{itemize}
\end{itemize}

CS 2340 bridges from academia to industry.


\end{frame}
%------------------------------------------------------------------------

\subsection{Tools and Technologies}

%------------------------------------------------------------------------
\begin{frame}[fragile]{"Pro" Java}

\begin{itemize}
\item The classpath
\item Project directory layout
\item Packages
\item Jar files
\item Build automation
\item Using an IDE
\end{itemize}


\end{frame}
%------------------------------------------------------------------------

%------------------------------------------------------------------------
\begin{frame}[fragile]{Web Applications}

\begin{itemize}
\item The HTTP protocol
\item Clients and Servers
\item Java Servlets and JSPs
\item Java web application servers
\end{itemize}

\end{frame}
%------------------------------------------------------------------------

\subsection{Agile Software Development}

%------------------------------------------------------------------------
\begin{frame}[fragile]{Software Engineering}

\begin{itemize}
\item Software development life cycle
\item Waterfall process models
\item Iterative process models
\item Methods for software design, implementation, and testing
\end{itemize}


\end{frame}
%------------------------------------------------------------------------

%------------------------------------------------------------------------
\begin{frame}[fragile]{Agile Development}

Agile Practices
\begin{itemize}
\item Pair programming
\item Clean code
\item Unit testing
\item Simple design
\item Refactoring
\end{itemize}
Agile project management (Scrum)
\begin{itemize}
\item Team roles
\item User stories
\item Small releases
\item Estimation
\end{itemize}


\end{frame}
%------------------------------------------------------------------------

\subsection{Object-oriented Design}

%------------------------------------------------------------------------
\begin{frame}[fragile]{Software Design}

\begin{itemize}
\item Design principles
\item Design techniques
\item System architectures
\item Design documentation
\end{itemize}


\end{frame}
%------------------------------------------------------------------------

%------------------------------------------------------------------------
\begin{frame}[fragile]{Object-Oriented Design}

\begin{itemize}
\item {\bf S}ingle Responsibility Principle
\item {\bf O}pen Closed Principle
\item {\bf L}iskov Substitution Principle
\item {\bf I}nterface Segregation Principle
\item {\bf D}ependency Inversion Principle
\end{itemize}

\end{frame}
%------------------------------------------------------------------------

%------------------------------------------------------------------------
\begin{frame}[fragile]{Design Patterns}

\begin{columns}[c]

\begin{column}{2in}
\begin{center}
\includegraphics[width=1.9in]{design-patterns-book.png}
\end{center}
\end{column}

\begin{column}{3in}
A recurring object-oriented design.
\begin{itemize}
\item Make proven techniques more accessible to developers of new systems -- don't have to study other systems.
\item Helps in choosing designs that make the system more reusable.
\item Facilitate documenentation and communication with other developers.
\end{itemize}
Design pattern catalog: descriptions of communicating objects and classes that are customized to solve a general design problem in a particular context.

\end{column}

\end{columns}

\end{frame}
%------------------------------------------------------------------------

% %------------------------------------------------------------------------
% \begin{frame}[fragile]{}


% \begin{lstlisting}[language=Python]

% \end{lstlisting}


% \end{frame}
% %------------------------------------------------------------------------

\end{document}
