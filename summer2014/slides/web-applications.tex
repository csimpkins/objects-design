\documentclass{beamer}

\newcommand{\course}{CS 2340 Objects and Design}
\newcommand{\lesson}{Web Applications}
\newcommand{\code}{http://www.cc.gatech.edu/~simpkins/teaching/gatech/cs2340/code}

\author[Chris Simpkins] 
{Christopher Simpkins \\\texttt{chris.simpkins@gatech.edu}}
\institute[Georgia Tech] % (optional, but mostly needed)

\date[CS 1331]{}

\subject{\lesson}


% If you have a file called "university-logo-filename.xxx", where xxx
% is a graphic format that can be processed by latex or pdflatex,
% resp., then you can add a logo as follows:

% \pgfdeclareimage[width=0.6in]{coc-logo}{cc_2012_logo}
% \logo{\pgfuseimage{coc-logo}}

\mode<presentation>
{
  \usetheme{Berlin}
  \useoutertheme{infolines}

  % or ...

 \setbeamercovered{transparent}
  % or whatever (possibly just delete it)
}

\usepackage{hyperref}
\usepackage{fancybox}
\usepackage{listings}
\usepackage[abbr]{harvard}

\usepackage[english]{babel}
% or whatever

\usepackage[latin1]{inputenc}
% or whatever

\usepackage{times}
\usepackage[T1]{fontenc}
% Or whatever. Note that the encoding and the font should match. If T1
% does not look nice, try deleting the line with the fontenc.


\usepackage{listings}
 
% "define" Scala
\lstdefinelanguage{scala}{
  morekeywords={abstract,case,catch,class,def,%
    do,else,extends,false,final,finally,%
    for,if,implicit,import,match,mixin,%
    new,null,object,override,package,%
    private,protected,requires,return,sealed,%
    super,this,throw,trait,true,try,%
    type,val,var,while,with,yield},
  otherkeywords={=>,<-,<\%,<:,>:,\#,@},
  sensitive=true,
  morecomment=[l]{//},
  morecomment=[n]{/*}{*/},
  morestring=[b]",
  morestring=[b]',
  morestring=[b]""",
}

\usepackage{color}
\definecolor{dkgreen}{rgb}{0,0.6,0}
\definecolor{gray}{rgb}{0.5,0.5,0.5}
\definecolor{mauve}{rgb}{0.58,0,0.82}
 
% Default settings for code listings
\lstset{frame=tb,
  language=scala,
  aboveskip=2mm,
  belowskip=2mm,
  showstringspaces=false,
  columns=flexible,
  basicstyle={\scriptsize\ttfamily},
  numbers=none,
  numberstyle=\tiny\color{gray},
  keywordstyle=\color{blue},
  commentstyle=\color{dkgreen},
  stringstyle=\color{mauve},
  frame=single,
  breaklines=true,
  breakatwhitespace=true,
  keepspaces=true
  %tabsize=3
}


\title[\course] % (optional, use only with long
                                      % paper titles)
{\lesson}

\subtitle{}
%% {Include Only If Paper Has a Subtitle}

\newcommand{\link}[2]{\href{#1}{\textcolor{blue}{\underline{#2}}}}


% \beamerdefaultoverlayspecification{<+->}


\begin{document}

\begin{frame}
  \titlepage
\end{frame}


%------------------------------------------------------------------------
\begin{frame}[fragile]{Web Applications}


\begin{itemize}
\item Web Applications
\item Apache Tomcat
\item First Java Web App
\end{itemize}

\end{frame}
%------------------------------------------------------------------------

%------------------------------------------------------------------------
\begin{frame}[fragile]{Web Applications}


A web application is client-server application that uses  the hyper-text transfer protocol (HTTP).
\begin{itemize}
\item HTTP request is sent from client to server
\item HTTP response is sent back to client from server
\item HTTP is stateless - there is no inherent relationship betwen request/response pairs
\begin{itemize}
\item We simulate sessions (related request/response pairs) by setting cookies on the client.
\end{itemize}
\end{itemize}
Web browsers -- Firefox, Chrome -- are platforms for clients.  Web servers -- Apache, Tomcat, nginx -- are plaforms for servers.  A particular set of web pages running in a browser that communicate with a particular set of web server applications constitutes a web application.

\end{frame}
%------------------------------------------------------------------------

%------------------------------------------------------------------------
\begin{frame}[fragile]{HTTP Protocol}


HTTP request message contain a request line, headers, and a body.  Each request line specifies a method.  Methods we care about:
\begin{itemize}
\item GET - get a resource from a server running at a specified URI
\item POST
\item UPDATE
\item DELETE
\end{itemize}
For example, if you type {\tt http://www.gatech.edu/} in your browser's address bar, or follow a hyperlink whose target is {\tt http://www.gatech.edu/}, you browser will send a GET request that looks something like this:
\begin{lstlisting}[language=Java]
GET http://www.gatech.edu/ HTTP/1.1
\end{lstlisting}
By the way, the inclusion of the access mechanism http:// makes the URI above a URL.  In gneral, though, it's a waste of mentons to distinguish between URIs and URLs.\\

For details see \link{http://www.w3.org/Protocols/rfc2616/rfc2616-sec5.html}{http://www.w3.org/Protocols/rfc2616/rfc2616-sec5.html}

\end{frame}
%------------------------------------------------------------------------

%------------------------------------------------------------------------
\begin{frame}[fragile]{Web App Structure}


Web applications can be arbitrarily rich, but the core functionality of most web applications is to manage resources by implementing four operations:

\begin{itemize}
\item Create - create a new instance of a resourece (new email message, new customer account object, etc) - maps to the HTTP POST method.
\item Read - read a resource - maps to the HTTP GET method.
\item Update - modify a resource - maps to the HTTP PUT method.
\item Delete - delete a resource - maps to the HTTP DELETE method.
\end{itemize}
This paradigm is called ``CRUD'' and most web frameworks (and RESTful web services) are structured around these operations.  In our sample application we'll see a simple way to map these operations to HTTP methods

\end{frame}
%------------------------------------------------------------------------

%------------------------------------------------------------------------
\begin{frame}[fragile]{Tomcat and Sample Application}


Now let's 
\begin{itemize}
\item download, install and configure Tomcat, and
\item discuss a simple web application usign Java servlets and JSPs. 
\end{itemize}


\end{frame}
%------------------------------------------------------------------------

% %------------------------------------------------------------------------
% \begin{frame}[fragile]{}


% \begin{lstlisting}[language=Java]

% \end{lstlisting}

% \begin{itemize}
% \item
% \end{itemize}


% \end{frame}
% %------------------------------------------------------------------------


\end{document}
